
Prolog~\cite{Colmerauer:1993:BP:154766.155362} is one of the first logic programming languages to become available, yet it still one the most popular logic programming language in use today. Prolog is a \emph{backwards-chaining} logic programming language where a program is composed of a set of rules that can be activated by inputing a query. Given a query $q(\hat{x})$, a Prolog interpreter will work backwards by matching $q(\hat{x})$ against the head of a rule. If found, the interpreter will then try to match the body of the rule, recursively, until it finds the program axioms (rules without body). If the search procedure succeeds, the interpreter finds a valid substitution for the $\hat{x}$ variables.

Datalog~\cite{Ramakrishnan93asurvey} is very popular logic programming language for deductive databases. Although it is a syntactic subset of Prolog, it works differently than Prolog because it is a \emph{forward-chaining} logic programming language. In Datalog, the program is composed of a database of facts $D$ and a set of rules $R$. Initially, the database $D$ starts with a set of axioms given by the program. Then, the interpreter will pick a subset of rules $R'$ of $R$ that can be applied using $D$. By applying $R'$, new facts are derived and added to the database.

Datalog interpreters usually employ a technique called \emph{Iterative Fixpoint Evaluation} (IFE) or \emph{semi-na\"{i}ve evaluation}~\cite{Ramakrishnan93asurvey} in order to generate the complete set of facts from the derivation rules. In IFE, computation is performed in iterations. In the first iteration, we consider all the axioms and
apply all rules that can be applied using the database. Once the database will be updated with the newly derived facts, we apply again all the rules given
the new database. If a rule is applied with at-least one fact that was derived in the previous iteration, then the facts instantiated by the head of the
rule can be added to the database. Otherwise, we ignore them since such facts must already been derived before. This is repeated until no new facts are added to the database.

IFE takes advantage of the fact that Datalog is based on Horn clauses~\cite{journals/jsyml/Horn51} and is thus monotonic. When a rule is applied, new information is added to the database but not removed. The database can be implemented efficiently since facts are stored there until the program completes.

We have designed a new forward-chaining logic programming language called Linear Meld (LM). LM is similar to Datalog but uses linear logic~\cite{girard-87} as its foundation. Since linear logic is the \emph{logic of resources}, LM rules may consume facts from the database, which makes IFE impossible to use as
an evaluation strategy. LM is concurrent because the forward-chaining proof process is done at several places - called \emph{nodes} - and derivation of facts may lead to communication between them. The communication paths between nodes and the nodes themselves create a graph data structure where computation happens concurrently. LM also includes novel features such as comprehensions and aggregates, which allow the programmer to iterate over
or accumulate values from the database, respectively.

In this paper, we want to present an high level dynamic semantics and then a low level dynamic semantics of LM.
Both semantics will be presented using proof-theoretic methods. The high level semantics will be very closely related to the linear logic sequent calculus.
The low level semantics will be much closer to a real implementation since it removes most of non-determinism of the high level semantics and
also describes in detail how the proof search mechanism works, including backtracking and resource management. We also relate both semantics by
proving that the low level dynamic semantics is sound in relation to the high level dynamic semantics.

The paper is organized as follows. In the first section, we give a brief overview of LM, including its syntax and how it works in practice.
Next, we present the connection between LM and linear logic, including the connectives used and how they relate to LM programs. We then present
the high level and low level dynamic semantics, followed by the soundness proof. Finally, we finish the paper with some conclusions.