
LM differs greatly from other Datalog-like languages due to the use of linear logic~\cite{Girard95logic:its}. Traditional forward-chaining logic programming languages make only use of classical logic, in which derived facts are true forever. Many ad-hoc extensions~\cite{Liu98extendingdatalog,Ludascher95alogical} have been devised in the past to support state updates in Datalog, but most are extra-logical which makes it harder to reason about programs.
LM uses linear logic as its foundation, therefore state updates are completely natural to the language.

We use a small subset of the original linear logic proof system along with an extension for definitions to improve
the expressiveness of the language. We summarize the connectives used in Table~\ref{table:linear}
and how they relate to LM.

\begin{table*}
   \begin{center}
\resizebox{16cm}{!}{
    \begin{tabular}{ | l | l || l | l | l |}
    \hline
    Connective                   & Description                                      & LM Syntax                                  & LM Place     & LM Example                                  \\ \hline \hline
    $\emph{fact}(\hat{x})$       & Linear facts.                                    & $fact(\hat{x})$                               & Body or Head    & \texttt{path(A, P)}                            \\ \hline
    $\bang \emph{fact}(\hat{x})$ & Persistent facts.                                & $\bang fact(\hat{x})$                         & Body or Head    & \texttt{$\bang$edge(X, Y, W)}                  \\ \hline
    $1$                          & Represents rules with an empty head.             & $1$                                           & Head            & \texttt{1}                                     \\ \hline
    $A \otimes B$                & Connect two expressions.                         & $A, B$                                        & Body and Head   & \texttt{path(A, P), edge(A, B, W)}             \\ \hline
    $\forall x. A$               & To represent variables defined inside the rule.  & Please see $A \lolli B$                       & Rule            & \texttt{path(A, B) $\lolli$ reachable(A, B)}   \\ \hline
    $\exists x. A$               & Instantiates new node variables.                 & $exists \; \widehat{x}. (B)$                  & Head            & \texttt{exists A.(path(A, P))}                 \\ \hline
    $A \lolli B$                 & $\lolli$ means "linearly implies".               & $A \lolli B$                                  & Rule            & \texttt{path(A, B) $\lolli$ reachable(A, B)}   \\
                                 & $A$ is the body and $B$ is the head.             &                                               &                 &                                                \\ \hline
    $\m{def} A. B$               & Extension called definitions.                    & $\{\; \widehat{x} \; | \; A \; | \; B \; \}$  & Head            & \texttt{\{B | !edge(A, B) | visit(B)\}}        \\
                                 & Used for comprehensions and aggregates.          &                                               &                 &                                                \\ \hline
    \end{tabular}
}
\end{center}
\caption{Connectives from Linear Logic used in LM.}
\label{table:linear}
\end{table*}

The sequent calculus for our linear logic fragment is shown in Fig.~\ref{linear_logic}.
The sequent has the form $\Psi ; \seqnocut{\Gamma}{\Delta}{C}$, where $\Psi$ is the typed
term context used in the quantifiers, $\Gamma$ is the multi-set of persistent terms, $\Delta$
is the multi-set of linear terms and $C$ is the term we want to prove.

\begin{figure}[ht]
\[
\infer[\one R]
{\Psi ; \seqnocut{\Gamma}{\cdot}{\one}}
{}
\tab
\infer[\one L]
{\Psi ; \seqnocut{\Gamma}{\Delta, \one}{C}}
{\Psi ; \seqnocut{\Gamma}{\Delta}{C}}
\]

\[
\infer[\with R]
{\Psi ; \seqnocut{\Gamma}{\Delta}{A \with B}}
{\Psi ; \seqnocut{\Gamma}{\Delta}{A} & \seqnocut{\Gamma}{\Delta}{B}}
\tab
\infer[\with L_1]
{\Psi ; \seqnocut{\Gamma}{\Delta, A \with B}{C}}
{\Psi ; \seqnocut{\Gamma}{\Delta, A}{C}}
\tab
\infer[\with L_2]
{\Psi ; \seqnocut{\Gamma}{\Delta, B \with B}{C}}
{\Psi ; \seqnocut{\Gamma}{\Delta, B}{C}}
\]

\[
\infer[\otimes R]
{\Psi ; \seqnocut{\Gamma}{\Delta, \Delta'}{A \otimes B}}
{\Psi ; \seqnocut{\Gamma}{\Delta}{A} & \seqnocut{\Gamma}{\Delta}{B}}
\tab
\infer[\otimes L]
{\Psi ;\seqnocut{\Gamma}{\Delta, A \otimes B}{C}}
{\Psi ; \seqnocut{\Gamma}{\Delta, A, B}{C}}
\]

\[
\infer[\lolli R]
{\Psi ; \seqnocut{\Gamma}{\Delta}{A \lolli B}}
{\Psi ; \seqnocut{\Gamma}{\Delta, A}{B}}
\tab
\infer[\lolli L]
{\seqnocut{\Gamma}{\Delta, \Delta', A \lolli B}{C}}
{\Psi ; \seqnocut{\Gamma}{\Delta}{A} &
   \Psi ; \seqnocut{\Gamma}{\Delta', B}{C}}
\]

\[
\infer[\forall R]
{\Psi ; \seqnocut{\Gamma}{\Delta}{\forall n:\tau. A}}
{\Psi, m:\tau ; \seqnocut{\Gamma}{\Delta}{A\{m/n\}}}
\tab
\infer[\forall L]
{\Psi ; \seqnocut{\Gamma}{\Delta, \forall n:\tau. A}{C}}
{\Psi \vdash M : \tau & \Psi ; \seqnocut{\Gamma}{\Delta, A\{M/n\}}{C}}
\]

\[
\infer[\exists R]
{\Psi ; \seqnocut{\Gamma}{\Delta}{\exists n: \tau. A}}
{\Psi \vdash M : \tau &
   \Psi ; \seqnocut{\Gamma}{\Delta}{A \{M/n\}}}
\tab
\infer[\exists L]
{\Psi ; \seqnocut{\Gamma}{\Delta, \exists n:\tau. A}{C}}
{\Psi, m:\tau ; \seqnocut{\Gamma}{\Delta, A\{m/n\}}{C}}
\]

\[
\infer[\bang R]
{\Psi ; \seqnocut{\Gamma}{\cdot}{\bang A}}
{\Psi ; \seqnocut{\Gamma}{\cdot}{A}}
\tab
\infer[\bang L]
{\Psi ; \seqnocut{\Gamma}{\Delta, \bang A}{C}}
{\Psi ; \seqnocut{\Gamma, A}{\Delta}{C}}
\tab
\infer[\m{copy}]
{\Psi ; \seqnocut{\Gamma, A}{\Delta}{C}}
{\Psi ; \seqnocut{\Gamma, A}{\Delta, A}{C}}
\]

\[
\infer[\m{def} \; R]
{\Psi ; \seqnocut{\Gamma}{\Delta}{\compr{A'}}}
{\Psi ; \seqnocut{\Gamma}{\Delta}{B\theta} &
 A \defeq B & A' \doteq A\theta}
\tab
\infer[\m{def} \; L]
{\Psi ; \seqnocut{\Gamma}{\Delta, \compr{A'}}{C}}
{
   \Psi ; \seqnocut{\Gamma}{\Delta, B\theta}{C} & A \defeq B & A' \doteq A\theta
}
\]
\caption{Linear logic fragment used in LM.}\label{linear_logic}
\end{figure}

Most connectives in our fragment are standard and well known, except for the $\compr{A}$ connective. This
connective is a definition that can be unfolded recursively and is used to logically justify
comprehensions and aggregates. We took inspiration from Baelde's work on least and greatest fixed points
in linear logic~\cite{Baelde:2012:LGF:2071368.2071370}. Baelde's system goes beyond simple recursive
definitions and allows proofs using induction and co-induction in linear logic. Fortunately,
the proof system satisfies cut-elimination which makes it consistent.

\begin{samepage}
In a comprehension, we want to apply an implication as many times as possible matches we can do
using the current database. One way to formally describe comprehensions would be to use persistent
rules that would be used a few times and then would be forgotten. A more reasonable approach is to use
definitions. Given a comprehension $C~=~\{~\widehat{x}~|~A~|~B~\}$ with a body $A$ and a head $B$, then we can build the following definition:
\nopagebreak
\[
\compr{C} \defeq 1 \with ((A \lolli B) \otimes \compr{C})
\]
\end{samepage}

We can unfold $\compr{C}$ to either stop (by selecting $1$) or get a new linear implication $A \lolli B$
to apply the comprehension once. Because we also get another $\compr{C}$ by selecting the right hand side,
the comprehension can be applied again, recursively.

\begin{samepage}
Aggregates work identically, but they need an extra argument to accumulate the aggregated value. If a sum aggregate $C$ has the form $[~\m{sum}~\Rightarrow~y~|~\widehat{x}~|~A~|~B_1~|~B_2~]$, then the definition will be as follows:

\nopagebreak
\[
\compr{C} \; V \defeq (\lambda v. B_2)V \with (\forall x. ((Ax \lolli B_1) \otimes \compr{C} \; (x + V)))
\]
\nopagebreak

The aggregate is initiated as $\compr{C} \; 0$.
\end{samepage}