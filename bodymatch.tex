The soundness theorem will be proven into two main steps. First, we prove that performing a rule match at LLD is sound in relation to HLD and
then we prove that the derivation of head terms in LLD is also sound.

In order to prove the soundness of matching, we want to reconstitute a valid $\mz$in HLD from a valid $\mo$in LLD. However, in LLD
we may fail during matching, therefore our body match lemma needs to handle unsuccessful matches. In order to be able to use induction, we
must assume a matching proposition $\mo$that already contains some continuation frames in stack $C$ that is well-formed in relation to the rule's body $A$ and initial database.

Our lemma also needs to apply to our continuation judgment $\contlld$, because when inverting some of
the matching assumptions, we get a continuation proposition. Apart from an unsuccessful match, we deal
with two situations during a successful match: (1) we succeed without needing to backtrack to a frame
in stack $C$ or (2) we need to backtrack to a frame in $C$. The complete lemma is stated and proven below.

\begin{lemma}[Body match]\label{thm:body_match}
   
Given a rule body $A$, consider a triplet $T = A; \Gamma; \Delta_{N}$ and a context $\Delta_{N} = \Delta_1, \Delta_2, \Xi$.

If $\mo \Gamma; \Delta_1, \Delta_2; \Xi; \Omega; H; C; R \rightarrow \Xi'; \Delta'; \Gamma'$ is well-formed in relation to $T$ then either:

\begin{enumerate}
   \item Match fails: \\ $\cont \cdot; H; R; \Gamma \rightarrow \Xi'; \Delta'; \Gamma'$
   \item Match succeeds: \\
      $\mz \Gamma; \Delta_x \rightarrow A$ (where $\Delta_x \subseteq \Delta_N$) and one of the three sub-cases is true:
      \begin{enumerate}
         \item No backtracking to frames of stack $C$: \\ $\mo \Gamma; \Delta_1; \Xi, \Delta_2; \cdot; H; C'', C; R \rightarrow \Xi'; \Delta'; \Gamma'$ (well-formed in relation to $T$) and $\Delta_x = \Xi, \Delta_2$
         \item With backtracking to a linear continuation frame: \\ $\exists f = (\Delta_a; \Delta_{b_1}, p_2, \Delta_{b_2}; p; \Omega_1, ..., \Omega_k; \Xi_1, ..., \Xi_m; \Lambda_1, ..., \Lambda_m; \Upsilon_1, ..., \Upsilon_n) \in C$ where $C = C', f, C''$ and $f$ turns into $f' = (\Delta_a, \Delta_{b_1}, p_2; \Delta_{b_2}; p; \Omega_1, ..., \Omega_k; \Xi_1, ..., \Xi_m; \Lambda_1, ..., \Lambda_m; \Upsilon_1, ..., \Upsilon_n)$ such that:
         \begin{enumerate}
            \item $\mo \Gamma; \Delta_c; \Xi_1, ..., \Xi_m, p_2, \Xi_c; \cdot; H; C''', f', C''; R \rightarrow \Xi'; \Delta'; \Gamma'$ (well-formed in relation to $T$) where $\Delta_c = (\Delta_1, \Delta_2, \Xi) - (\Xi_1, ..., \Xi_m, p_2, \Xi_c)$
         \end{enumerate}
         \item With backtracking to a persistent continuation frame: \\$\exists f = [\Gamma_1, p_2, \Gamma_2; \Delta_{c_1}, \Delta_{c_2}; \Xi_c; \bang p; \Omega_1, ..., \Omega_k; \Lambda_1, ..., \Lambda_m; \Upsilon_1, ..., \Upsilon_n] \in C$ where $C = C', f, C''$ and $f$ turns into $f' = [\Gamma_2; \Delta_{c_1}, \Delta_{c_2}; \Xi_1, ..., \Xi_m; \bang p; \Omega_1, ..., \Omega_k; \Lambda_1, ..., \Lambda_m; \Upsilon_1, ..., \Upsilon_n]$ such that:
         \begin{enumerate}
            \item $\mo \Gamma; \Delta_{c_1}; \Xi_1, ..., \Xi_m, \Delta_{c_2}; \cdot; H; C'', f', C''; R \rightarrow \Xi'; \Delta'; \Gamma'$ (well-formed in relation to $T$) where $\Delta_{c_1}, \Delta_{c_2} = (\Delta_1, \Delta_2, \Xi) - (\Xi_1, ..., \Xi_m)$
         \end{enumerate}
      \end{enumerate}
\end{enumerate}

If $\cont C; H; R; \Gamma \rightarrow \Xi'; \Delta'; \Gamma'$ and $C$ is well-formed in relation to $T$ then either:

\begin{enumerate}
   \item Match fails: \\ $\cont \cdot; H; R; \Gamma \rightarrow \Xi'; \Delta'; \Gamma'$
   \item Match succeeds: \\ $\mz \Delta_x \rightarrow A$ (where $\Delta_x \subseteq \Delta_N$) where one sub-case is true:
   \begin{enumerate}
      \item With backtracking to a linear continuation frame: \\ $\exists f = (\Delta_a; \Delta_{b_1}, p_2, \Delta_{b_2}; \Xi_1, ..., \Xi_m; p; \Omega_1, ..., \Omega_k; \Lambda_1, ..., \Lambda_m; \Upsilon_1, ..., \Upsilon_n) \in C$ where $C = C', f, C''$ and $f$ turns into $f' = (\Delta_a, \Delta_{b_1}, p_2; \Delta_{b_2}; p; \Omega_1, ..., \Omega_k; \Xi_1, ..., \Xi_m; \Lambda_1, ..., \Lambda_m; \Upsilon_1, ..., \Upsilon_n)$ such that:
      \begin{enumerate}
         \item $\mo \Gamma; \Delta_c; \Xi_1, ..., \Xi_m, p_2, \Xi_c; \cdot; H; C''', f', C''; R \rightarrow \Xi'; \Delta'; \Gamma'$ (well-formed in relation to $T$) where $\Delta_c = (\Delta_1, \Delta_2, \Xi) - (\Xi_1, ..., \Xi_m, p_2, \Xi_c)$ and $\Delta_x = \Xi_1, ..., \Xi_m, p_2, \Xi_c$
      \end{enumerate}
      \item With backtracking to a persistent continuation frame: \\$\exists f = [\Gamma_1, p_2, \Gamma_2; \Delta_{c_1}, \Delta_{c_2}; \Xi_c; \bang p; \Omega_1, ..., \Omega_k; \Lambda_1, ..., \Lambda_m; \Upsilon_1, ..., \Upsilon_n] \in C$ where $C = C', f, C''$ and $f$ turns into $f' = [\Gamma_2; \Delta_{c_1}, \Delta_{c_2}; \Xi_1, ..., \Xi_m; \bang p; \Omega_1, ..., \Omega_k; \Lambda_1, ..., \Lambda_m; \Upsilon_1, ..., \Upsilon_n]$ such that:
      \begin{enumerate}
         \item $\mo \Gamma; \Delta_{c_1}; \Xi_1, ..., \Xi_m, \Delta_{c_2}; \cdot; H; C'', f', C''; R \rightarrow \Xi'; \Delta'; \Gamma'$ (well-formed in relation to $T$) where $\Delta_{c_1}, \Delta_{c_2} = \Delta_1, \Delta_2, \Xi - (\Xi_1, ..., \Xi_m)$ and $\Delta_x = \Xi_1, ..., \Xi_m, \Delta_{c2}$
      \end{enumerate}
   \end{enumerate}
\end{enumerate}
\end{lemma}

\begin{proof}
   Proof by mutual induction. In $\mo$on the size of $\Omega$ and on $\contlld$, first on the size of the second argument of the frame ($\Delta''$ and $\Gamma''$) and then on the size of the stack $C$. When inverting the assumption, the well-formedness of the stack and match are proven straightforwardly using the well-formedness of the assumption and the match equivalence theorem.
\end{proof}

For the induction hypothesis to be applicable in in Lemma~\ref{thm:body_match} there must be
a relation between the judgments $\mo$and $\contlld$.
We can define a lexicographic ordering $A \prec B$, meaning that proposition $A$ has a smaller proof than proposition $B$ (potentially $A$ is sub-proof of $B$),
or alternatively, $A$ is "executed later" than $B$.
The specific ordering is as follows:

\begin{enumerate}
   \item $\cont C; H; R; \Gamma \rightarrow \Xi'; \Delta'; \Gamma' \prec \cont C', C; H; R; \Gamma \rightarrow \Xi'; \Delta'; \Gamma'$\\
   The continuation must use the top of the stack $C'$ before using $C$;
   \item $\cont C', (\Delta, \Delta_1; \Delta_2; \Xi; p; \Omega; \Lambda; \Upsilon), C; H; R; \Gamma \rightarrow \Xi'; \Delta'; \Gamma'$\\
   \hspace*{1cm}$\prec \cont C'', (\Delta; \Delta_1, \Delta_2; \Xi; p; \Omega; \Lambda; \Upsilon), C; H; R; \Gamma \rightarrow \Xi'; \Delta'; \Gamma'$\\
   A continuation frame with more candidates has more steps to do than a frame with less candidates;
   \item $\cont C', [\Gamma_1; \Delta; \Xi; \bang p; \Omega; \Lambda; \Upsilon], C; H; R; \Gamma \rightarrow \Xi'; \Delta'; \Gamma'$\\
   \hspace*{1cm} $\prec \cont C'', [\Gamma_2, \Gamma_3; \Delta; \Xi; \bang p; \Omega; \Lambda; \Upsilon], C; H; R; \Gamma \rightarrow \Xi'; \Delta'; \Gamma'$\\
      Same as the previous one;
   \item $\cont C; H; R; \Gamma \rightarrow \Xi'; \Delta'; \Gamma' \prec \mo \Gamma; \Delta; \Xi; \Omega; H; C', C; R \rightarrow \Xi'; \Delta'; \Gamma'$\\
   Same as (1);
   \item $\mo \Gamma; \Delta; \Xi; \Omega; H; C; R \rightarrow \Xi'; \Delta'; \Gamma' \prec \cont C', C; H; R; \Gamma \rightarrow \Xi'; \Delta'; \Gamma'$\\
   Same as the previous one;
   \item $\mo \Gamma; \Delta''; \Xi''; \Omega'; H; C', C; R \rightarrow \Xi'; \Delta'; \Gamma' \prec \mo \Gamma; \Delta; \Xi; \Omega; H; C; R \rightarrow \Xi'; \Delta'; \Gamma'$ as long as $\Omega' \prec \Omega$\\
   Adding continuation frames to the stack makes the proof smaller as long as $\Omega$ is also smaller; 
   \item $\mo \Gamma; \Delta; \Xi; \Omega; H; C', (\Delta, \Delta_1; \Delta_2; \Xi; p; \Omega; \Lambda; \Upsilon), C; R \rightarrow \Xi'; \Delta'$\\
   \hspace*{1cm} $\prec \mo \Gamma; \Delta''; \Xi''; \Omega'; C'', (\Delta; \Delta_1, \Delta_2; \Xi; p; \Omega; \Lambda; \Upsilon), C; R \rightarrow \Xi'; \Delta'; \Gamma'$\\
   Same as (2);
   \item $\mo \Gamma; \Delta; \Xi; \Omega; H; C', [\Gamma_1; \Delta; \Xi; \bang p; \Omega; \Lambda; \Upsilon], C; R \rightarrow \Xi'; \Delta'; \Gamma'$\\
   \hspace*{1cm} $\prec \mo \Gamma; \Delta''; \Xi''; \Omega'; C'', [\Gamma_2, \Gamma_3; \Delta; \Xi; \bang p; \Omega; \Lambda; \Upsilon], C; R \rightarrow \Xi'; \Delta'; \Gamma'$\\
   Same as (3).
\end{enumerate}
