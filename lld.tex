The low level dynamic semantics (LLD) removes all the non-deterministic choices in HLD
and makes them deterministic. The new semantics will do the following:

\begin{itemize}
   \item Match rules by priority order;
   \item Determine the set of linear facts needed to match either the body of the rule or the body of comprehensions without guessing;
   \item Apply as many comprehensions as the database allows;
   \item Apply as many aggregates as the database allows.
\end{itemize}

HLD had many different proof trees for a given triplet $\Gamma; \Delta; \Phi$ because HLD allows choices to be made during the inference rules. For instance, in HLD any rule could be selected to be executed. In LLD there is only one proof tree
for a given $\Gamma; \Delta; \Phi$ since there is no guessing involved. LLD semantics present a complete step by step mechanism that is needed to correctly evaluate an LM program. For instance, when LLD tries to apply a rule, it will check if there is enough
facts in the database and backtrack until a rule can be applied.

\begin{theorem}
In LLD, there is only one possible proof for a given $\Gamma; \Delta; \Phi$.
\end{theorem}

\begin{proof}
Inference rules of every judgment in LLD are disjunct, therefore only one inference rule can be applied at any given point in the proof.
\end{proof}

LLD shares exactly the same inputs and outputs as HLD. The first difference between the two systems starts when picking a rule to derive.
Instead of guessing, LLD treats the list of rules as a stack and picks the first rule $R$ to execute (the rule with the highest priority). The remaining rules are stored as a continuation. If $R$ cannot be matched because there is not enough facts in the database,
we backtrack and use the rule continuation to pick the next rule and so on, until one rule can be successfully applied.

\subsection{Continuation Frames}

The most interesting aspect introduced by LLD is the continuation frame. A continuation frame acts as a choice
point that is created during rule matching whenever we try to match a fact expression against the database.
The frame considers all the facts relevant to the expression given the current variable bindings and predicate,
that may or not fail during the remaining matching process.

The frame contains enough state to resume the matching
process at the time of its creation, therefore we can easily backtrack to the choice point and select the next candidate
fact from the database (judgment $\contlld$ or $\contclld$).
We keep all the continuation frames in a continuation stack, so in order to backtrack we update the top frame and
select the next fact candidate to continue the matching process.
If all candidates are exhausted, we pop the top frame and try the next one.

By using this match mechanism, we can determine which facts need to be used to match a rule.
Our LM implementation works like LLD, by iterating over
the available facts at each choice point and then committing to the rule if the matching process
succeeds. However, while the implementation only attempts to match rules with a very high change of success,
LLD is more na\"{i}ve in this aspect because it tries all rules in order.

\subsection{Structure of Continuation Frames}

We have two continuation frame types. One frame is stored when dealing with persistent fact expressions and the other when dealing
with linear fact expressions.

\subsubsection{Persistent Frame}

A persistent frame has the form $[\Gamma'; \Delta; \Xi; \bang p; \Omega; \Lambda; \Upsilon]$, where:

\begin{enumerate}
   \item[$\Delta$] remaining multi-set of linear facts;
   \item[$\Xi$] multi-set of linear facts we have consumed to reach this point;
   \item[$\bang p$] current fact expression that originated this choice point;
   \item[$\Omega$] ordered list of remaining terms needed to match past this choice point;
   \item[$\Lambda$] multi-set of linear fact expressions that we have matched to reach this choice point. All the linear facts that match these terms are located in $\Xi$;
   \item[$\Upsilon$] multi-set of persistent fact expressions that we matched up to this point.
\end{enumerate}

\subsubsection{Linear Frame}

A linear frame has the form $(\Delta; \Delta'; \Xi; p; \Omega; \Lambda; \Upsilon)$, where:

\begin{enumerate}
   \item[$\Delta$] multi-set of linear facts that are not of type $p$ plus all the other $p$'s we have already tried, including the current $p$;
   \item[$\Delta'$] all the other $p$'s we haven't tried yet. It is a multi-set of linear facts;
   \item[$\Xi$] multi-set of linear facts we have consumed to reach this point;
   \item[$p$] current fact expression that originated this choice point;
   \item[$\Omega$] ordered list of remaining terms needed to match past this choice point;
   \item[$\Lambda$] multi-set of linear fact expressions that we have matched to reach this choice point. All the linear facts that match these terms are located in $\Xi$;
   \item[$\Upsilon$] multi-set of persistent fact expressions that we matched up to this point.
\end{enumerate}

\subsection{Step}

The starting inference rule of LLD is identical to the one used in HLD described in \ref{step_hld}.

\[
\infer[\stepz]
{\stepo [\Gamma_1 .. \mid \Gamma_i .. \mid \Gamma_n]; [\Delta_1 .. \mid \Delta_i .. \mid \Delta_n]; \Phi \Longrightarrow [\Gamma_1, \Gamma'_1 .. \mid \Gamma_i, \Gamma'_i .. \mid \Gamma_n, \Gamma'_n]; [\Delta_1, \Delta'_1 .. \mid (\Delta_i - \Xi'), \Delta'_i .. \mid \Delta_n, \Delta'_n]}
{\doo \Gamma_i; \Delta_i; \Phi \rightarrow \Xi'; \Delta'_1, ..., \Delta'_n; \Gamma'_1, ..., \Gamma'_n}
\]

\subsection{Application}

When starting an execution step, we pick the first rule $R_1$ and create a rule continuation with the form $(\Phi, \Delta)$, where $\Phi$
is the stack of remaining rules and $\Delta$ is the starting linear context. Context $\Gamma$ does not need to be saved
because its facts are never deleted.

\[
\infer[\doo \m{rule}]
{\doo \Gamma; \Delta; R_1, \Phi \rightarrow \Xi'; \Delta';\Gamma'}
{\ao \Gamma; \Delta; R_1; (\Phi, \Delta) \rightarrow \Xi';\Delta';\Gamma'}
\]

\[
\infer[\ao \m{rule}]
{\ao \Gamma; \Delta; A \lolli B; R \rightarrow \Xi'; \Delta'; \Gamma'}
{\mo \Gamma; \Delta; \cdot; A; B; \cdot; R \rightarrow \Xi'; \Delta'; \Gamma'}
\]

\subsection{Match}\label{sec:body_match}

The matching judgment uses the form $\mo \Gamma; \Delta; \Xi; \Omega; H; C; R \rightarrow \Xi'; \Delta'; \Gamma'$ where:

\begin{enumerate}
   \item[$\Delta$] multi-set of linear facts still available to complete the matching process;
   \item[$\Xi$] multi-set of linear facts consumed up to this point;
   \item[$\Omega$] ordered list of deconstructed head terms to match;
   \item[$H$] head of the rule;
   \item[$C$] ordered list of frames representing the continuation stack;
   \item[$R$] rule continuation to be used if the current rule fails;
   \item[$\Xi'$] multi-set of consumed linear facts after applying the rule;
   \item[$\Delta'$] multi-set of derived linear facts after applying the rule;
   \item[$\Gamma'$] multi-set of derived persistent facts after applying the rule.
\end{enumerate}

Matching will attempt to use facts from $\Delta$ and $\Gamma$ to match the terms of the body of the rule represented as $\Omega$. During this process continuation frames are pushed into $C$.
If the matching process fails, we use the continuation stack through the $\cont$judgment.

\subsubsection{Linear fact expressions}

The first 4 inference rules are used when the head of $\Omega$ is a linear fact expression $p$: (1) the
continuation stack is empty; (2) the continuation is not empty and a linear continuation frame is at the top;
(3) the continuation stack is not empty and a persistent continuation frame is at the top;
and (4) we cannot find any linear fact in the database that matches $p$. Note that in the first 3 rules,
we find $p_1$ and $\Delta''$ as facts from the database that match $p$ the hidden and partially initialized arguments.
Context $\Delta''$ is stored in the second argument of the new continuation frame but passes along with $\Delta$ since they
have not been consumed yet (just $p_1$).

{\footnotesize
\[
\infer[\mo p~\m{first}]
{\mo \Gamma ; \Delta, p_1, \Delta'' ; \Xi; p, \Omega; H; \cdot; R \rightarrow \Xi'; \Delta'; \Gamma'}
{p_1, \Delta'' \prec p & \mo \Gamma ; \Delta, \Delta''; \Xi, p_1; \Omega; H; (\Delta, p_1; \Delta''; p; \Omega; \Xi; \cdot; \cdot); R \rightarrow \Xi'; \Delta'; \Gamma'}
\]

\[
\infer[\mo p~\m{on}~q]
{\mo \Gamma ; \Delta, p_1, \Delta'' ; \Xi; p, \Omega; H; C_1, C; R \rightarrow \Xi'; \Delta'; \Gamma'}
{p_1, \Delta'' \prec p & \begin{split} C_1 &= (\Delta_{old}; \Delta'_{old}; q; \Omega_{old}; \Xi_{old}; \Lambda; \Upsilon) \\ \mo &\Gamma ; \Delta, \Delta''; \Xi, p_1; \Omega; H; (\Delta, p_1; \Delta''; p; \Omega; \Xi; q, \Lambda; \Upsilon), C_1, C; R \rightarrow \Xi'; \Delta'; \Gamma'\end{split}}
\]


\[
\infer[\mo p~\m{on}~\bang q]
{\mo \Gamma ; \Delta, p_1, \Delta'' ; \Xi; p, \Omega; H; C_1, C; R \rightarrow \Xi'; \Delta'; \Gamma'}
{p_1, \Delta'' \prec p & \begin{split}C_1 &= [\Gamma_{old}; \Delta_{old}; \bang q; \Omega_{old}; \Xi_{old}; \Lambda; \Upsilon] \\ \mo &\Gamma ; \Delta, \Delta''; \Xi, p_1; \Omega; H; (\Delta, p_1; \Delta''; p; \Omega; \Xi; \Lambda; q, \Upsilon), C_1, C; R \rightarrow \Xi'; \Delta'; \Gamma' \end{split}}
\]

\[
\infer[\mo p~\m{fail}]
{\mo \Gamma ; \Delta; \Xi; p, \Omega; H; C; R \rightarrow \Xi'; \Delta'; \Gamma'}
{\Delta \npreceq p & \cont C ; H; R; \Gamma \rightarrow \Xi'; \Delta'; \Gamma'}
\]
}

\subsubsection{Persistent fact expressions}

The next 4 inference rules are used when the head of $\Omega$ contains a persistent fact expression $\bang p$. They are identical
to the previous 4 rules but they deal with the persistent context $\Gamma$.

{\footnotesize
\[
\infer[\mo \bang p~\m{first}]
{\mo \Gamma, p_1, \Gamma'' ; \Delta; \Xi; \bang p, \Omega; H; \cdot; R \rightarrow \Xi'; \Delta'; \Gamma'}
{p_1, \Gamma'' \prec \bang p & \mo \Gamma, p_1, \Gamma'' ; \Delta; \Xi; \Omega; H; [\Gamma''; \Delta; \bang p ; \Omega; \Xi; \cdot; \cdot]; R \rightarrow \Xi'; \Delta'; \Gamma'}
\]

\[
\infer[\mo \bang p~\m{on}~q]
{\mo \Gamma, p_1, \Gamma'' ; \Delta; \Xi; \bang p, \Omega; H; C_1, C; R \rightarrow \Xi'; \Delta'; \Gamma'}
{p_1, \Gamma'' \prec \bang p & \begin{split}C_1 &= (\Delta_{old}; \Delta'_{old}; q; \Omega_{old}; \Xi_{old}; \Lambda; \Upsilon) \\ \mo &\Gamma, p_1, \Gamma'' ; \Delta; \Xi; \Omega; H; [\Gamma''; \Delta; \bang p ; \Omega; \Xi; q, \Lambda; \Upsilon], C_1, C; R \rightarrow \Xi'; \Delta'; \Gamma'\end{split}}
\]


\[
\infer[\mo \bang p~\m{on}~\bang q]
{\mo \Gamma, p_1, \Gamma'' ; \Delta; \Xi; \bang p, \Omega; H; C_1, C; R \rightarrow \Xi'; \Delta'; \Gamma'}
{p_1, \Gamma'' \prec \bang p & \begin{split}C_1 &= [\Gamma_{old}; \Delta_{old}; \bang q; \Omega_{old}; \Xi_{old}; \Lambda; \Upsilon] \\ \mo &\Gamma, p_1, \Gamma'' ; \Delta; \Xi; \Omega; H; [\Gamma''; \Delta; \bang p ; \Omega; \Xi; \Lambda; q, \Upsilon], C_1, C; R \rightarrow \Xi'; \Delta'; \Gamma'\end{split}}
\]

\[
\infer[\mo \bang p~\m{fail}]
{\mo \Gamma ; \Delta; \Xi; \bang p, \Omega; H; C; R \rightarrow \Xi'; \Delta'; \Gamma'}
{\Gamma \npreceq \bang p & \cont C; H; R; \Gamma \rightarrow \Xi'; \Delta'; \Gamma'}
\]
}

\subsubsection{Other expressions}

Finally, we have the inference rule $\mo \otimes$ that deconstructs body terms with $\otimes$ and the $\mo \m{end}$ rule
that terminates the matching process and initiates the derivation of head terms because $\Omega$ is empty.

{\footnotesize
\[
\infer[\mo \otimes]
{\mo \Gamma ; \Delta; \Xi; A \otimes B, \Omega ; H ; C; R \rightarrow \Xi'; \Delta';\Gamma'}
{\mo \Gamma ; \Delta; \Xi; A, B, \Omega; H; C; R \rightarrow \Xi';\Delta';\Gamma'}
\tab
\infer[\mo \m{end}]
{\mo \Gamma ; \Delta; \Xi; \cdot ; H; C; R \rightarrow \Xi'; \Delta'; \Gamma'}
{\done \Gamma ; \Delta; \Xi; \cdot ; H; \cdot \rightarrow \Xi'; \Delta'; \Gamma'}
\]
}

\subsection{Continuation}\label{sec:match_cont}

If the previous matching process fails, we pick the top frame from the stack $C$ and restore the matching process using another fact and/or context. The continuation judgment $\cont C; H; R; \Gamma \rightarrow \Xi'; \Delta'; \Gamma'$ deals with the backtracking process where the meaning of each argument is as follows:

\begin{enumerate}
   \item[$C$] continuation stack;
   \item[$H$] head of the current rule we are executing;
   \item[$R$] next available rules if the current rule does not match.
\end{enumerate}

\subsubsection{Linear continuation frames}

The next two inference rules detect a linear continuation frame on top of the continuation stack. The first rule picks the next
fact candidate $p_1$ from the second argument of the linear frame and updates the frame. Otherwise, if we have no more
candidate facts, we pop the continuation frame and use $\cont$again with the remaining continuation stack.

%{\footnotesize
\[
\infer[\cont p~\m{next}]
{\cont (\Delta; p_1, \Delta''; p, \Omega; \Xi; \Lambda; \Upsilon), C; H; R; \Gamma \rightarrow \Xi'; \Delta'; \Gamma'}
{\mo \Gamma ; \Delta, \Delta''; \Xi, p_1; \Omega; H; (\Delta, p_1; \Delta''; p, \Omega; H; \Xi; \Lambda; \Upsilon), C; R \rightarrow \Xi'; \Delta'; \Gamma'}
\]
%}

%{\footnotesize
\[
\infer[\cont p~\m{no~more}]
{\cont (\Delta; \cdot; p, \Omega; \Xi; \Lambda; \Upsilon), C; H; R; \Gamma \rightarrow \Xi'; \Delta'; \Gamma'}
{\cont C; H; R; \Gamma; \Xi'; \Delta'; \Gamma'}
\]
%}

\subsubsection{Persistent continuation frames} We also have the same two kinds of inference rules for persistent continuation frames.

%{\footnotesize
\[
\infer[\cont \bang p~\m{next}]
{\cont [p_1, \Gamma'; \Delta; \bang p, \Omega; \Xi; \Lambda; \Upsilon], C; H; R; \Gamma \rightarrow \Xi'; \Delta'; \Gamma'}
{\mo \Gamma; \Delta; \Xi; \Omega; H; [\Gamma'; \Delta; \bang p, \Omega; \Xi; \Lambda; \Upsilon], C; R \rightarrow \Xi'; \Delta'; \Gamma'}
\]
%}

%{\footnotesize
\[
\infer[\cont \bang p~\m{no~more}]
{\cont [\cdot; \Delta; \bang p, \Omega; \Xi; \Lambda; \Upsilon], C; H; R; \Gamma \rightarrow \Xi'; \Delta'; \Gamma'}
{\cont C; H; R; \Gamma \rightarrow \Xi'; \Delta'; \Gamma'}
\]
%}

\subsubsection{Empty continuation stack}

Finally, if the continuation stack is empty, we simply force execution to try the next inference rule in $\Phi$.

%{\footnotesize
\[
\infer[\cont \m{next~rule}]
{\cont \cdot; H; (\Phi, \Delta); \Gamma \rightarrow \Xi'; \Delta'; \Gamma'}
{\doo \Gamma; \Delta; \Phi \rightarrow \Xi'; \Delta'; \Gamma'}
\]
%}

\subsection{Derivation}

Once the list of terms $\Omega$ in the $\mo$judgment is exhausted (rule $\mo \m{end}$), we derive the terms of the head of rule.
The derivation judgment uses the form $\done \Gamma; \Delta; \Xi; \Gamma_1; \Delta_1; \Omega \rightarrow \Xi'; \Delta'; \Gamma'$,
where:

\begin{enumerate}
   \item[$\Delta$] multi-set of linear facts we started with minus the linear facts consumed $\Xi$ during the matching of the body of the rule;
   \item[$\Xi$] multi-set of linear facts consumed during the matching of the body of the rule;
   \item[$\Gamma_1$] set of persistent facts derived up to this point in the derivation;
   \item[$\Delta_1$] multi-set of linear facts derived up to this point in the derivation;
   \item[$\Omega$] remaining terms to derive as an ordered list. We start with $B$ if the original rule is $A \lolli B$.
\end{enumerate}

\subsubsection{Fact templates} When deriving either $p$ or $\bang p$ we have the following two inference rules:

{\footnotesize
\[
\infer[\done p]
{\done \Gamma ; \Delta; \Xi; \Gamma_1 ; \Delta_1; p, \Omega \rightarrow \Xi'; \Delta'; \Gamma'}
{\done \Gamma ; \Delta; \Xi; \Gamma_1 ; p, \Delta_1; \Omega \rightarrow \Xi'; \Delta'; \Gamma'}
\tab
\infer[\done \bang p]
{\done \Gamma ; \Delta ; \Xi; \Gamma_1 ; \Delta_1; \bang p, \Omega \rightarrow \Xi'; \Delta'; \Gamma'}
{\done \Gamma ; \Delta ; \Xi; \Gamma_1, p; \Delta_1; \Omega \rightarrow \Xi'; \Delta'; \Gamma'}
\]
}

\subsubsection{Deconstruct head}
The following two inference rules deconstruct the head list $\Omega$ from terms created using either $1$ or $\otimes$.

{\footnotesize
\[
\infer[\done 1]
{\done \Gamma; \Delta; \Xi; \Gamma_1 ; \Delta_1; 1, \Omega \rightarrow \Xi';\Delta';\Gamma'}
{\done \Gamma; \Delta; \Xi; \Gamma_1 ; \Delta_1; \Omega \rightarrow \Xi'; \Delta';\Gamma'}
\tab
\infer[\done \otimes]
{\done \Gamma ; \Delta; \Xi; \Gamma_1; \Delta_1; A \otimes B, \Omega \rightarrow \Xi'; \Delta';\Gamma'}
{\done \Gamma ; \Delta; \Xi; \Gamma_1; \Delta_1; A, B, \Omega \rightarrow \Xi';\Delta';\Gamma'}
\]
}

\subsubsection{Comprehensions} The inference rule $\done \m{comp}$ deals with comprehensions. Comprehensions have the form $A \com B$, where $A$ is the body and $B$ the head of the comprehension. Judgment $\mc$initiates the matching process of the body $A$
of the comprehension (explained in the next section).

{\footnotesize
\[
\infer[\done \m{comp}]
{\done \Gamma; \Delta ; \Xi; \Gamma_1; \Delta_1; A \com B, \Omega \rightarrow \Xi';\Delta';\Gamma'}
{\mc \Gamma; \Delta; \Xi; \Gamma_1; \Delta_1; \cdot; A ; B ; \cdot; \cdot; A \com B; \Omega; \Delta \rightarrow \Xi';\Delta';\Gamma'}
\]
}

\subsubsection{Successful rule} Finally, if the ordered list $\Omega$ is exhausted, then the whole execution process is done.
Note how the output arguments match the input arguments of the $\done$judgment.

{\footnotesize
\[
\infer[\done \m{end}]
{\done \Gamma; \Delta; \Xi; \Gamma_1; \Delta_1; \cdot \rightarrow \Xi; \Delta_1; \Gamma_1}
{}
\]
}


\subsection{Comprehension Match}

The matching process for comprehensions is similar to the process used for matching the body of the rule presented
in Section~\ref{sec:body_match}, however we use two continuation stacks, $C$ and $P$. In $P$, we put all the initial persistent frames
and in $C$ we put the first linear frame and then everything else.

To apply as many comprehensions as the database allows, we
reuse the continuation stacks to backtrack and try all the possible combinations available.
However, in order to reuse the stack, we need to update it by removing all the frames pushed after the first linear continuation frame.
If we tried to use those frames, we would assumed that the linear facts used by the other frames were still in the database, but that
is not true because they have been consumed during the first application of the comprehension.
For example, if the body is $\bang \mathtt{a(X)} \otimes \mathtt{b(X)} \otimes \mathtt{c(X)}$ and the continuation stack has three frames (one per fact), we cannot backtrack to the frame of $\mathtt{c(X)}$
since, at that point, the matching process was assuming that the previous \texttt{b(X)} linear fact was still available.
Moreover, we also need to remove the consumed linear facts from the frames of \texttt{b(X)} and $\bang$\texttt{a(X)} in order to make the stack fully consistent with the new database. We will see later on how to do that.

The full match judgment has the form
$\mc \Gamma; \Delta; \Xi_N; \Gamma_{N1}; \Delta_{N1}; \Xi; \Omega; C; P; AB; \Omega_N; \Delta_N \rightarrow \Xi'; \Delta'; \Gamma'$, where:

\begin{enumerate}
   \item[$\Delta$] multi-set of linear facts remaining up to this point in the matching process;
   \item[$\Xi_N$] multi-set of linear facts used during the matching process of the body of the rule and all the previous comprehensions;
   \item[$\Gamma_{N1}$] set of persistent facts derived up to this point in the head of the rule;
   \item[$\Delta_{N1}$] multi-set of linear facts derived up to this point in the head of the rule;
   \item[$\Xi$] multi-set of linear facts consumed up to this point;
   \item[$\Omega$] ordered list of terms that need to be matched for the comprehension to be applied;
   \item[$C$] continuation stack that contains both linear and persistent frames. The first frame must be linear;
   \item[$P$] initial part of the continuation stack with only persistent frames;
   \item[$AB$] comprehension $A \com B$ that is being matched;
   \item[$\Omega_N$] ordered list of remaining terms of the head of the rule to be derived;
   \item[$\Delta_N$] multi-set of linear facts that were still available after matching the body of the rule and all the previous comprehensions. Note that $\Delta, \Xi = \Delta_N$.
\end{enumerate}

\subsubsection{Linear continuation frames}

The following five inference rules deal with the case when there is a linear fact expression in the body of the comprehension.

{\scriptsize
\[
\infer[\mc p~\m{first}]
{\mc \Gamma; \Delta, p_1, \Delta''; \Xi_N; \Gamma_{N1}; \Delta_{N1}; \cdot; p, \Omega; \cdot; \cdot; AB; \Omega_N; \Delta_N \rightarrow \Xi'; \Delta'; \Gamma'}
{p_1, \Delta'' \prec p & \mc \Gamma; \Delta, \Delta''; \Xi_N; \Gamma_{N1}; \Delta_{N1}; \Xi, p_1; \Omega; (\Delta, p_1; \Delta''; \cdot; p; \Omega; \cdot; \cdot); \cdot; AB; \Omega_N; \Delta_N \rightarrow \Xi'; \Delta'; \Gamma'}
\]

\[
\infer[\mc p~\m{on}~q]
{\mc \Gamma; \Delta, p_1, \Delta''; \Xi_N; \Gamma_{N1}; \Delta_{N1}; \Xi; p, \Omega; C_1, C; P; AB; \Omega_N; \Delta_N \rightarrow \Xi'; \Delta'; \Gamma'}
{p_1, \Delta'' \prec p & \begin{split}C_1 =& (\Delta_{old}; \Delta'_{old}; \Xi_{old}; q; \Omega_{old}; \Lambda; \Upsilon) \\ \mc &\Gamma; \Delta, \Delta''; \Xi_N; \Gamma_{N1}; \Delta_{N1}; \Xi, p_1; \Omega; (\Delta, p_1; \Delta''; \Xi; p; \Omega; q, \Lambda; \Upsilon), C_1, C; P; AB; \Omega_N; \Delta_N \rightarrow \Xi'; \Delta'; \Gamma'\end{split}}
\]


\[
\infer[\mc p~\m{on}~\bang q~C]
{\mc \Gamma; \Delta, p_1, \Delta''; \Xi_N; \Gamma_{N1}; \Delta_{N1}; \Xi; p, \Omega; C_1, C; P; AB; \Omega_N; \Delta_N \rightarrow \Xi'; \Delta'; \Gamma'}
{p_1, \Delta'' \prec p & \begin{split}C_1 &= [\Gamma_{old}; \Delta_{old}; \Xi_{old}; q; \Omega_{old}; \Lambda; \Upsilon] \\ \mc &\Gamma; \Delta, \Delta''; \Xi_N; \Gamma_{N1}; \Delta_{N1}; \Xi, p_1; \Omega; (\Delta, p_1; \Delta''; \Xi; p; \Omega; \Lambda; q, \Upsilon), C_1, C; P; AB; \Omega_N; \Delta_N \rightarrow \Xi'; \Delta'; \Gamma' \end{split}}
\]


\[
\infer[\mc p~\m{on}~\bang q~P]
{\mc \Gamma; \Delta, p_1, \Delta''; \Xi_N; \Gamma_{N1}; \Delta_{N1}; \cdot; p, \Omega; \cdot; P_1, P; AB; \Omega_N; \Delta_N \rightarrow \Xi'; \Delta'; \Gamma'}
{p_1, \Delta'' \prec p & \begin{split}\\ P_1 &= [\Gamma_{old}; \Delta_N; \cdot; q; \Omega_{old}; \cdot; \Upsilon]\\ \Delta_N &= \Delta, p_1, \Delta'' \\ \mc &\Gamma; \Delta, \Delta''; \Xi_N; \Gamma_{N1}; \Delta_{N1}; p_1; \Omega; (\Delta, p_1; \Delta''; \cdot; p; \Omega; \cdot; q, \Upsilon); P_1, P; AB; \Omega_N; \Delta_N \rightarrow \Xi'; \Delta'; \Gamma'\end{split}}
\]


\[
\infer[\mc p~\m{fail}]
{\mc \Gamma; \Delta; \Xi_N; \Gamma_{N1}; \Delta_{N1}; \Xi; p, \Omega; C; P; AB; \Omega_N; \Delta_N \rightarrow \Xi'; \Delta'; \Gamma'}
{\Delta \npreceq p & \contc \Gamma; \Delta_N; \Xi_N; \Gamma_{N1}; \Delta_{N1}; C; P; AB; \Omega_N \rightarrow \Xi'; \Delta'; \Gamma'}
\]
}

\subsubsection{Persistent continuation frames}

If we find a persistent fact expression, the inference rules are similar to the previous ones.

{\scriptsize
\[
\infer[\mc \bang p~\m{first}]
{\mc \Gamma, p_1, \Gamma''; \Delta_N; \Xi_N; \Gamma_{N1}; \Delta_{N1}; \cdot; \bang p, \Omega; \cdot; \cdot; AB; \Omega_N; \Delta_N \rightarrow \Xi'; \Delta'; \Gamma'}
{p_1, \Gamma'' \prec \bang p & \mc \Gamma, p_1, \Gamma''; \Delta_N; \Xi_N; \Gamma_{N1}; \Delta_{N1}; \cdot; \Omega; \cdot; [\Gamma''; \Delta_N; \cdot; \bang p; \cdot; \Omega; \cdot; \cdot]; AB; \Omega_N; \Delta_N \rightarrow \Xi'; \Delta'; \Gamma'}
\]

\[
\infer[\mc \bang p~\m{on}~\bang q~P]
{\mc \Gamma, p_1, \Gamma''; \Delta_N; \Xi_N; \Gamma_{N1}; \Delta_{N1}; \cdot; \bang p, \Omega; \cdot; P_1, P; AB; \Omega_N; \Delta_N \rightarrow \Xi'; \Delta'; \Gamma'}
{p_1, \Gamma'' \prec \bang p & \begin{split}P_1 &= [\Gamma_{old}; \Delta_N; \cdot; \bang q; \Omega_{old}; \cdot; \Upsilon] \\ \mc &\Gamma, p_1, \Gamma''; \Delta_N; \Xi_N; \Gamma_{N1}; \Delta_{N1}; \cdot; \Omega; [\Gamma''; \Delta_N; \cdot; \bang p; \cdot; \Omega; \cdot; q, \Upsilon], P_1, P; AB; \Omega_N; \Delta_N \rightarrow \Xi'; \Delta'; \Gamma'\end{split}}
\]


\[
\infer[\mc \bang p~\m{on}~\bang q~C]
{\mc \Gamma, p_1, \Gamma''; \Delta; \Xi_N; \Gamma_{N1}; \Delta_{N1}; \Xi; \bang p, \Omega; C_1, C; P; AB; \Omega_N; \Delta_N \rightarrow \Xi'; \Delta'; \Gamma'}
{p_1, \Gamma'' \prec \bang p & \begin{split}C_1 &= [\Gamma_{old}; \Delta_{old}; \Xi_{old}; \bang q; \Omega_{old}; \Lambda; \Upsilon] \\ \mc & \Gamma, p_1, \Gamma''; \Delta; \Xi_N; \Gamma_{N1}; \Delta_{N1}; \Xi; \Omega; [\Gamma''; \Delta; \Xi; \bang p; \cdot; \Omega; \Lambda; q, \Upsilon], C_1, C; P; AB; \Omega_N; \Delta_N \rightarrow \Xi'; \Delta'; \Gamma' \end{split}}
\]


\[
\infer[\mc \bang p~\m{on}~q~C]
{\mc \Gamma, p_1, \Gamma''; \Delta; \Xi_N; \Gamma_{N1}; \Delta_{N1}; \Xi; \bang p, \Omega; C_1, C; P; AB; \Omega_N; \Delta_N \rightarrow \Xi'; \Delta'; \Gamma'}
{p_1, \Gamma'' \prec \bang p & \begin{split}C_1 &= (\Delta_{old}; \Delta'_{old}; \Xi_{old}; q; \Omega_{old}; \Lambda; \Upsilon) \\ \mc &\Gamma, p_1, \Gamma''; \Delta; \Xi_N; \Gamma_{N1}; \Delta_{N1}; \Xi; \Omega; [\Gamma''; \Delta; \Xi; \bang p; \cdot; \Omega; \Lambda, q; \Upsilon], C_1, C; P; AB; \Omega_N; \Delta_N \rightarrow \Xi'; \Delta'; \Gamma'\end{split}}
\]

\[
\infer[\mc \bang p~\m{fail}]
{\mc \Gamma; \Delta; \Xi_N; \Gamma_{N1}; \Delta_{N1}; \Xi; \bang p, \Omega; C; P; AB; \Omega_N; \Delta_N \rightarrow \Xi'; \Delta; \Gamma'}
{\Gamma \npreceq \bang p & \contc \Gamma; \Delta_N; \Xi_N; \Gamma_{N1}; \Delta_{N1}; C; P; AB; \Omega_N \rightarrow \Xi'; \Delta'; \Gamma'}
\]
}

\subsubsection{Deconstruct body}

The inference rules for deconstructing the body of the comprehension, including the connectives $\otimes$ and $1$, are shown next.

{\scriptsize
\[
\infer[\mc \otimes]
{\mc \Delta; \Xi_N; \Delta_{N1}; \Xi; X \otimes Y, \Omega; C; P; AB; \Omega_N; \Delta_N \rightarrow \Xi'; \Delta'}
{\mc \Delta; \Xi_N; \Delta_{N1}; \Xi; X, Y, \Omega; C; P; AB; \Omega_N; \Delta_N \rightarrow \Xi'; \Delta'}
\]

\[
\infer[\mc 1]
{\mc \Delta; \Xi_N; \Delta_{N1}; \Xi; 1, \Omega; C; P; AB; \Omega_N; \Delta_N \rightarrow \Xi'; \Delta'}
{\mc \Delta; \Xi_N; \Delta_{N1}; \Xi; \Omega; C; P; AB; \Omega_N; \Delta_N \rightarrow \Xi'; \Delta'}
\]
}

\subsubsection{Successful match}

Finally, when there is no more body terms to process, we initiate the process of the updating the continuation stack (with $\dall$) followed
by deriving the head of the comprehension.

{\scriptsize
\[
\infer[\mc \m{end}]
{\mc \Gamma; \Delta; \Xi_N; \Gamma_{N1}; \Delta_{N1}; \Xi; \cdot; C; P; AB; \Omega_N; \Delta_N \rightarrow \Xi'; \Delta'; \Gamma'}
{\dall \Gamma; \Xi_N; \Gamma_{N1}; \Delta_{N1}; \Xi; C; P; AB; \Omega_N; \Delta_N \rightarrow \Xi'; \Delta'; \Gamma'}
\]
}

\subsection{Continuation Stack Update}

As we said before, to update the continuation stacks, we need remove to all the frames except the first linear frame and remove the consumed linear facts from the remaining frames so that they are still valid for the next application of the comprehension.
The judgment that updates the stack has the form $\dall \Gamma; \Xi_N; \Gamma_{N1}; \Delta_{N1}; \Xi; C; P; AB; \Omega_N; \Delta_N \rightarrow \Xi'; \Delta'; \Gamma'$, where:

\begin{enumerate}
   \item[$\Xi_N$] multi-set of linear facts used during the matching process of the body of the rule and all the previous comprehensions;
   \item[$\Gamma_{N1}$] set of persistent facts derived by the head of the rule and all the previous comprehensions;
   \item[$\Delta_{N1}$] multi-set of linear facts derived by the head of the rule and all the previous comprehensions;
   \item[$\Xi$] multi-set of linear facts consumed by the previous application of the comprehension;
   \item[$C, P$] continuation stacks for the comprehension;
   \item[$AB$] comprehension $A \com B$ that is being computed;
   \item[$\Omega_N$] ordered list of remaining terms of the head of the rule to be derived;
   \item[$\Delta_N$] multi-set of linear facts that were still available after matching the body of the rule and all the previous comprehensions.
\end{enumerate}

\subsubsection{Remove linear continuation frames}

To remove all linear continuation frames except the first one, we simply go through all the frames in the stack $C$ with $\dall \m{more}$
until only one frame remains. This frame is then updated using $\dall \m{end~linear}$ by removing the linear facts $\Xi$ consumed
during the last application of the comprehension.

{\footnotesize
\[
\infer[\dall \m{end~linear}]
{\dall \Gamma; \Xi_N; \Gamma_{N1}; \Delta_{N1}; \Xi; (\Delta_x; \Delta''; \cdot; p; \Omega; \cdot; \Upsilon); P;  AB; \Omega_N; \Delta_N \rightarrow \Xi'; \Delta'; \Gamma'}
{\begin{split}\strans &\Xi; P; P' \\ \dc \Gamma; \Xi_N, \Xi; \Gamma_{N1}; \Delta_{N1}; (\Delta_x - \Xi; \Delta'' - \Xi; \cdot;& p; \Omega; \cdot; \Upsilon) ; P' ; AB; \Omega_N; (\Delta_N - \Xi) \rightarrow \Xi'; \Delta'; \Gamma'\end{split}}
\]

\[
\infer[\dall \m{more}]
{\dall \Gamma; \Xi_N; \Gamma_{N1}; \Delta_{N1}; \Xi; \_, X, C; P; AB; \Omega_N; \Delta_N \rightarrow \Xi'; \Delta'; \Gamma'}
{\dall \Gamma; \Xi_N; \Gamma_{N1}; \Delta_{N1}; \Xi; X, C; P; AB; \Omega_N; \Delta_N \rightarrow \Xi'; \Delta'; \Gamma'}
\]

\[
\infer[\dall \m{end~empty}]
{\dall \Gamma; \Xi_N; \Gamma_{N1}; \Delta_{N1}; \Xi; \cdot; P;  AB; \Omega_N; \Delta_N \rightarrow \Xi'; \Delta'; \Gamma'}
{\begin{split}\strans &\Xi; P; P' \\ \dc \Gamma; \Xi_N, \Xi; \Gamma_{N1}; \Delta_{N1}; \cdot ; P' ; &AB; \Omega_N; (\Delta_N - \Xi) \rightarrow \Xi'; \Delta'; \Gamma'\end{split}}
\]
}

\subsubsection{Transform persistent continuation frames}

Finally, we also need to subtract the consumed facts in $\Xi$ from the second argument of every persistent continuation frame.
We have an auxiliary judgment $\strans \Xi; P; P'$ where $\Xi$ is the multi-set of consumed linear facts, $P$ is the persistent continuation
stack and $P'$ is the stack $P$ after the update.

{\footnotesize
\[
\infer[\strans]
{\strans \Xi; [\Gamma'; \Delta_N; \cdot; \bang p; \Omega; \cdot; \Upsilon], P; [\Gamma'; \Delta_N - \Xi; \cdot; \bang p, \Omega; \cdot; \Upsilon], P'}
{\strans \Xi; P; P'}
\tab
\infer[\strans \m{end}]
{\strans \Xi; \cdot; \cdot}
{\strans \Xi; \cdot; \cdot}
\]
}

\subsection{Comprehension Continuation}

If the comprehension match fails, we need to backtrack to the previous frame and restore the matching process at that point. The judgment used to backtrack is similar to the one presented in Section~\ref{sec:match_cont} and has the form $\contc \Gamma; \Delta_N; \Xi_N; \Delta_{N1}; C; P; AB; \Omega_N \rightarrow \Xi'; \Delta'; \Gamma'$, where:

\begin{enumerate}
   \item[$\Delta_N$] multi-set of linear facts still available after matching the body of the rule;
   \item[$\Xi_N$] multi-set of linear facts used to match the body of the rule and all the previous comprehensions;
   \item[$\Gamma_{N1}$] set of persistent facts derived by the head of the rule and all the previous comprehensions;
   \item[$\Delta_{N1}$] multi-set of linear facts derived by the head of the rule and all the previous comprehensions;
   \item[$C, P$] continuation stacks;
   \item[$AB$] comprehension $A \com B$ that is executing;
   \item[$\Omega_N$] ordered list of remaining terms of the head of the rule to be derived.
\end{enumerate}

\subsubsection{Using C stack}

The following 4 inference rules use the $C$ stack, the stack where the first continuation frame is linear, to perform backtracking.

{\small
\[
\infer[\contc \m{next}~C~p]
{\contc \Gamma; \Delta_N; \Xi_N; \Gamma_{N1}; \Delta_{N1}; (\Delta; p_1, \Delta''; \Xi; p; \Omega; \Lambda; \Upsilon), C; P; AB; \Omega_N \rightarrow \Xi'; \Delta'; \Gamma'}
{\mc \Gamma; \Delta; \Xi_N; \Gamma_{N1}; \Delta_{N1}; \Xi; \Omega; (\Delta, p_1; \Delta''; \Xi; p; \Omega; \Lambda; \Upsilon), C; P; AB; \Omega_N; \Delta_N \rightarrow \Xi'; \Delta'; \Gamma'}
\]

\[
\infer[\contc \m{next}~C~\bang p]
{\contc \Gamma; \Delta_N; \Xi_N; \Gamma_{N1}; \Delta_{N1}; [p_1, \Gamma'; \Delta; \Xi; \bang p; \Omega; \Lambda; \Upsilon], C; P; AB; \Omega_N \rightarrow \Xi'; \Delta'; \Gamma'}
{\mc \Gamma; \Delta; \Xi_N; \Gamma_{N1}; \Delta_{N1}; \Xi; \Omega; [\Gamma'; \Delta; \Xi; \bang p; \Omega; \Lambda; \Upsilon], C; P; AB; \Omega_N; \Delta_N \rightarrow \Xi'; \Delta'; \Gamma'}
\]

\[
\infer[\contc \m{next}~C~\m{empty}~p]
{\contc \Gamma; \Delta_N; \Xi_N; \Gamma_{N1}; \Delta_{N1}; (\Delta; \cdot; \Xi; p; \Omega; \Lambda; \Upsilon), C; P; AB; \Omega_N \rightarrow \Xi'; \Delta'; \Gamma'}
{\contc \Gamma; \Delta_N; \Xi_N; \Gamma_{N1}; \Delta_{N1}; C; P; AB; \Omega_N \rightarrow \Xi'; \Delta'; \Gamma'}
\]

\[
\infer[\contc \m{next}~C~\m{empty}~\bang p]
{\contc \Gamma; \Delta_N; \Xi_N; \Gamma_{N1}; \Delta_{N1}; [\cdot; \Delta; \Xi; \bang p; \Omega; \Lambda; \Upsilon], C; P; AB; \Omega_N \rightarrow \Xi'; \Delta'; \Gamma'}
{\contc \Gamma; \Delta_N; \Xi_N; \Gamma_{N1}; \Delta_{N1}; C; P; AB; \Omega_N \rightarrow \Xi'; \Delta'; \Gamma'}
\]
}

\subsubsection{Using P stack}

The following 2 inference rules use the $P$ stack instead, the stack where all continuation frames are persistent.

{\small
\[
\infer[\contc \m{next}~P~\bang p]
{\contc \Gamma; \Delta_N; \Xi_N; \Gamma_{N1}; \Delta_{N1}; \cdot; [p_1, \Gamma'; \Delta_N; \cdot; \bang p; \Omega; \cdot; \Upsilon], P; AB; \Omega_N \rightarrow \Xi'; \Delta'; \Gamma'}
{\mc \Gamma; \Delta_N; \Xi_N; \Gamma_{N1}; \Delta_{N1}; \cdot; \Omega; \cdot; [\Gamma'; \Delta_N; \cdot; \bang p; \Omega; \cdot; \Upsilon], P; AB; \Omega_N; \Delta_N \rightarrow \Xi'; \Delta'; \Gamma'}
\]

\[
\infer[\contc \m{next}~P~\m{empty}~\bang p]
{\contc \Gamma; \Delta_N; \Xi_N; \Gamma_{N1}; \Delta_{N1}; \cdot; [\cdot; \Delta_N; \cdot; \bang p; \Omega; \cdot; \Upsilon], P; AB; \Omega_N \rightarrow \Xi'; \Delta'; \Gamma'}
{\contc \Gamma; \Delta_N; \Xi_N; \Gamma_{N1}; \Delta_{N1}; \cdot; P; AB; \Omega_N \rightarrow \Xi'; \Delta'; \Gamma'}
\]
}

\subsubsection{Comprehension done}

If both the $C$ and $P$ stacks are empty, we are unable to backtrack and thus unable to continue deriving the comprehension.
We next derive the remaining head terms in $\Omega$.

{\small
\[
\infer[\contc \m{end}]
{\contc \Gamma; \Delta_N; \Xi_N; \Gamma_{N1}; \Delta_{N1}; \cdot; \cdot; AB; \Omega \rightarrow \Xi'; \Delta'; \Gamma'}
{\done \Gamma; \Delta_N; \Xi_N; \Gamma_{N1}; \Delta_{N1}; \Omega \rightarrow \Xi'; \Delta'; \Gamma'}
\]
}

\subsection{Comprehension Derivation}

After updating the continuation stacks, we then derive the head of the comprehension. The judgment has the form $\dc \Gamma; \Xi; \Gamma_1; \Delta_1; \Omega; C; P; AB; \Omega_N; \Delta_N \rightarrow \Xi'; \Delta'; \Gamma'$, where:

\begin{enumerate}
   \item[$\Xi$] multi-set of linear facts consumed both by the body of the rule and previous comprehension applications;
   \item[$\Gamma_1$] set of persistent facts derived by the head of the rule, previous comprehensions and current derivation;
   \item[$\Delta_1$] multi-set of linear facts derived by the head of the rule, previous comprehensions and current derivation;
   \item[$\Omega$] ordered list of terms to derive;
   \item[$C, P$] new continuation stacks;
   \item[$AB$] comprehension $A \com B$ that is being executed;
   \item[$\Omega_N$] ordered list of remaining terms of the head of the rule to be derived;
   \item[$\Delta_N$] multi-set of remaining linear facts that can be used for the next comprehension applications.
\end{enumerate}

{\footnotesize
\[
\infer[\dc p]
{\dc \Gamma; \Xi; \Gamma_1; \Delta_1; p, \Omega; C; P; AB; \Omega_N; \Delta_N \rightarrow \Xi'; \Delta'; \Gamma'}
{\dc \Gamma; \Xi; \Gamma_1; \Delta_1, p; \Omega; C; P; AB; \Omega_N; \Delta_N \rightarrow \Xi'; \Delta'; \Gamma'}
\]

\[
\infer[\dc \bang p]
{\dc \Gamma; \Xi; \Gamma_1; \Delta_1; \bang p, \Omega; C; P; AB; \Omega_N; \Delta_N \rightarrow \Xi'; \Delta'; \Gamma'}
{\dc \Gamma; \Xi; \Gamma_1, p; \Delta_1; \Omega; C; P; AB; \Omega_N; \Delta_N \rightarrow \Xi'; \Delta'; \Gamma'}
\]

\[
\infer[\dc \otimes]
{\dc \Gamma; \Xi; \Gamma_1; \Delta_1; A \otimes B, \Omega; C; P; AB; \Omega_N; \Delta_N \rightarrow \Xi'; \Delta';\Gamma'}
{\dc \Gamma; \Xi; \Gamma_1; \Delta_1; A, B, \Omega; C; P; AB; \Omega_N; \Delta_N \rightarrow \Xi'; \Delta';\Gamma'}
\]

\[
\infer[\dc 1]
{\dc \Gamma; \Xi; \Gamma_1; \Delta_1; 1, \Omega; C; P; AB; \Omega_N; \Delta_N \rightarrow \Xi'; \Delta';\Gamma'}
{\dc \Gamma; \Xi; \Gamma_1; \Delta_1; \Omega; C; P; AB; \Omega_N; \Delta_N \rightarrow \Xi'; \Delta';\Gamma'}
\]
}

The last derivation rule is activated whenever the $\Omega$ is empty. We use the $\contc$judgment to reuse the continuation stack
and try the next combination for the comprehension.

{\footnotesize
\[
\infer[\dc \m{end}]
{\dc \Gamma; \Xi; \Gamma_1; \Delta_1; \cdot; C; P; AB; \Omega_N; \Delta_N \rightarrow \Xi'; \Delta'; \Gamma'}
{\contc \Gamma; \Delta_N; \Xi; \Gamma_1; \Delta_1; C; P; AB; \Omega_N \rightarrow \Xi'; \Delta'; \Gamma'}
\]
}