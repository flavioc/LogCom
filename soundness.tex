
Now that we have presented both the HLD and LLD semantics, we are in position to start building our soundness theorem.
The soundness theorem proves that if a rule was successfully derived in the LLD semantics then it can also be derived
in the HLD semantics. Since the HLD semantics are so close to linear logic, we prove that our language has a determined, correct,
proof search behavior when executing programs. However, the completeness theorem cannot be proven since LLD lacks the non-determinism
inherent in HLD.


First and foremost, we need to prove some auxiliary theorems and definitions that will be used during the soundness theorem.

\subsection{Term Equivalence}
The first definition defines the equality between two multi-sets of terms.
Two multi-sets $A$ and $B$ are equal, $\feq{A}{B}$, when they have the same constituent atoms.

{\small
\[
\infer[\equiv p]
{\feq{p, A}{p, B}}
{\feq{A}{B}}
\tab
\infer[\equiv \bang p]
{\feq{\bang p, A}{\bang p, B}}
{\feq{A}{B}}
\tab
\infer[\equiv 1~L]
{\feq{1, A}{B}}
{\feq{A}{B}}
\tab
\infer[\equiv 1~R]
{\feq{A}{1, B}}
{\feq{A}{B}}
\]

\[
\infer[\equiv \cdot]
{\feq{\cdot}{\cdot}}
{}
\tab
\infer[\equiv \otimes~L]
{\feq{A \otimes B, C}{D}}
{\feq{A, B, C}{D}}
\tab
\infer[\equiv \otimes~R]
{\feq{A}{B \otimes C, D}}
{\feq{A}{B, C, D}}
\]
}

\begin{theorem}[Match equivalence]
If two multi-sets are equivalent, $\feq{A_1, ..., A_n}{B_1, ..., B_m}$, and we can match $A_1 \otimes ... \otimes A_n$ in HLD such that $\mz \Gamma ; \Delta \rightarrow A_1 \otimes ... \otimes A_n$ then $\mz \Gamma; \Delta \rightarrow B_1 \otimes ... \otimes B_m$ is also true.
\end{theorem}
\begin{proof}
By straightforward induction on the first assumption.
\end{proof}

\subsection{Well-Formed Continuation Frames}
Given a body $A$ that we want to match, the starting linear context $\Delta_{N}$ and the persistent context $\Gamma$, we want to
make sure that the frames used are always well-formed in relation to $A$ and the contexts.

\begin{definition}[Well-formed frame]

Given a triplet $A; \Gamma; \Delta_{N}$ where $A$ is a term, $\Gamma$ is a multi-set of persistent resources and $\Delta_{N}$ a multi-set of linear resources we say that a frame $f$ is well-formed iff:

\begin{enumerate}
   \item Linear frame $f = (\Delta, p_1; \Delta'; \Xi_1, ..., \Xi_m; p; \Omega_1, ..., \Omega_n; \Lambda_1, ..., \Lambda_m; \Upsilon_1, ..., \Upsilon_k)$
   \begin{enumerate}
      \item $\feq{p, \Omega_1, ..., \Omega_n, \Lambda_1, ..., \Lambda_m, \Upsilon_1, ..., \Upsilon_k}{A}$ (the remaining terms and already matched terms are equivalent to the initial body $A$);
      \item $\mz \Xi_1, ..., \Xi_m \rightarrow \Lambda_1 \otimes ... \otimes \Lambda_m$ and $\mz \Xi_i \rightarrow \Lambda_i$ for every $i$;
      \item $\Delta, \Delta', \Xi, p_1 = \Delta_{N}$ (available facts, candidate facts for $p$, consumed facts and the linear fact used for $p$, respectively, are the same as the initial $\Delta_{N}$);
      \item $\mz \Gamma; \cdot \rightarrow \Upsilon_1 \otimes ... \otimes \Upsilon_k$.
   \end{enumerate}
   \item Persistent frame $f = [\Gamma'; \Delta; \Xi_1, ..., \Xi_m; \bang p; \Omega_1, ..., \Omega_n; \Lambda_1, ..., \Lambda_m; \Upsilon_1, ..., \Upsilon_k]$
      \begin{enumerate}
         \item $\feq{\bang p, \Omega_1, ..., \Omega_n, \Lambda_1, ..., \Lambda_m, \Upsilon_1, ..., \Upsilon_k}{A}$;
         \item $\mz \Xi_1, ..., \Xi_m \rightarrow \Lambda_1 \otimes ... \otimes \Lambda_m$ and $\mz \Xi_i \rightarrow \Lambda_i$ for every $i$;
         \item $\Delta, \Xi = \Delta_{N}$;
         \item $\mz \Gamma; \cdot \rightarrow \bang p \otimes \Upsilon_1 \otimes ... \otimes \Upsilon_k$;
         \item $\Gamma' \subset \Gamma$ (remaining candidates are a subset of $\Gamma$).
      \end{enumerate}
\end{enumerate}
\end{definition}

\begin{definition}[Well-formed stack]
A continuation stack $C$ is well-formed iff every frame is well-formed.
\end{definition}

Given the previous definitions, we can now define what it means for a matching judgment to be well-formed.

\begin{definition}[Well-formed body match]

$\mo \Gamma; \Delta; \Xi; \Omega; H; C; R \rightarrow \Xi'; \Delta'; \Gamma'$ is well-formed in relation to a triplet $A; \Gamma; \Delta_{N}$ iff:

\begin{enumerate}
   \item $\Delta, \Xi = \Delta_{N}$
   \item $C$ is well-formed in relation to $A; \Gamma; \Delta_{N}$ and:
   \begin{enumerate}
      \item If $C = \cdot$
   
      $\feq{\Omega}{A}$.
   
      \item Else if $C = (\Delta_a, p_1; \Delta_b; \Xi''; p; \Omega'; \Lambda_1, ..., \Lambda_m; \Upsilon_1, ..., \Upsilon_k), C'$
   
      \begin{enumerate}
         \item $\feq{\Omega'}{\Omega}$;
         \item $p_1 \in \Xi$ and $\mz \Gamma; p_1 \rightarrow p$;
         \item $\Xi = \Xi'', p_1$;
         \item $\Delta = \Delta_a, \Delta_b$.
      \end{enumerate}
      \item Else if $C = [\Gamma'; \Delta''; \Xi''; \bang p; \Omega'; \Lambda_1, ..., \Lambda_m; \Upsilon_1, ..., \Upsilon_k], C'$
      \begin{enumerate}
         \item $\feq{\Omega}{\Omega'}$;
         \item $\Xi = \Xi''$;
         \item $\Delta = \Delta''$.
      \end{enumerate}
   \end{enumerate}
\end{enumerate}

\end{definition}

\begin{definition}[Well-formed comprehension match]
$\mc \Gamma; \Delta; \Xi_N; \Gamma_{N1}; \Delta_{N1}; \Xi; \Omega; C; P; A \com B; \Omega_N; \Delta_N \rightarrow \Xi'; \Delta'; \Gamma'$ is well-formed in relation to a triplet $A; \Gamma; \Delta_{N}$ iff:

\begin{enumerate}
   \item $P$ is only composed of persistent frames.
   \item $C$ is composed of either linear or persistent frames, but the first frame is linear.
   \item $\Delta, \Xi = \Delta_{N}$
   \item $C$ and $P$ are well-formed in relation to $A; \Gamma; \Delta_{N}$ and:
   \begin{enumerate}
      \item If $C = \cdot$ and $P = \cdot$
      
      $\feq{\Omega}{A}$.
      
      \item Else if $C = \cdot$ and $P = [\Gamma'; \Delta''; \Xi''; \bang p; \Omega'; \Lambda_1, ..., \Lambda_m; \Upsilon_1, ..., \Upsilon_k], P'$
      
      \begin{enumerate}
         \item $\feq{\Omega}{\Omega'}$;
         \item $\Xi = \Xi''$;
         \item $\Delta = \Delta''$.
      \end{enumerate}
      
      \item Else if $C = [\Gamma'; \Delta''; \Xi''; \bang p; \Omega'; \Lambda_1, ..., \Lambda_m; \Upsilon_1, ..., \Upsilon_k], C'$
      \begin{enumerate}
         \item $\feq{\Omega}{\Omega'}$;
         \item $\Xi = \Xi''$;
         \item $\Delta = \Delta''$.
      \end{enumerate}
      
      \item Else if $C = (\Delta_a, p_1; \Delta_b; \Xi''; p; \Omega'; \Lambda_1, ..., \Lambda_m; \Upsilon_1, ..., \Upsilon_k), C'$
   
      \begin{enumerate}
         \item $\feq{\Omega'}{\Omega}$;
         \item $p_1 \in \Xi$ and $\mz \Gamma; p_1 \rightarrow p$;
         \item $\Xi = \Xi'', p_1$;
         \item $\Delta = \Delta_a, \Delta_b$.
      \end{enumerate}
   \end{enumerate}
      
\end{enumerate}

\end{definition}


\subsection{Body Match Soundness}
The soundness theorem will be proven into two main steps. First, we prove that performing a rule match at LLD is sound in relation to HLD and
then we prove that the derivation of head terms in LLD is also sound.

In order to prove the soundness of matching, we want to reconstitute a valid $\mz$in HLD from a valid $\mo$in LLD. However, in LLD
we may fail during matching, therefore our body match lemma needs to handle unsuccessful matches. In order to be able to use induction, we
must assume a matching proposition $\mo$that already contains some continuation frames in stack $C$ that is well-formed in relation to the rule's body $A$ and initial database.

Our lemma also needs to apply to our continuation judgment $\contlld$, because when inverting some of
the matching assumptions, we get a continuation proposition. Apart from an unsuccessful match, we deal
with two situations during a successful match: (1) we succeed without needing to backtrack to a frame
in stack $C$ or (2) we need to backtrack to a frame in $C$. The complete lemma is stated and proven below.

\begin{lemma}[Body match]\label{thm:body_match}
   
Given a rule body $A$, consider a triplet $T = A; \Gamma; \Delta_{N}$ and a context $\Delta_{N} = \Delta_1, \Delta_2, \Xi$.

If $\mo \Gamma; \Delta_1, \Delta_2; \Xi; \Omega; H; C; R \rightarrow \Xi'; \Delta'; \Gamma'$ is well-formed in relation to $T$ then either:

\begin{enumerate}
   \item Match fails: \\ $\cont \cdot; H; R; \Gamma \rightarrow \Xi'; \Delta'; \Gamma'$
   \item Match succeeds: \\
      $\mz \Gamma; \Delta_x \rightarrow A$ (where $\Delta_x \subseteq \Delta_N$) and one of the three sub-cases is true:
      \begin{enumerate}
         \item No backtracking to frames of stack $C$: \\ $\mo \Gamma; \Delta_1; \Xi, \Delta_2; \cdot; H; C'', C; R \rightarrow \Xi'; \Delta'; \Gamma'$ (well-formed in relation to $T$) and $\Delta_x = \Xi, \Delta_2$
         \item With backtracking to a linear continuation frame: \\ $\exists f = (\Delta_a; \Delta_{b_1}, p_2, \Delta_{b_2}; p; \Omega_1, ..., \Omega_k; \Xi_1, ..., \Xi_m; \Lambda_1, ..., \Lambda_m; \Upsilon_1, ..., \Upsilon_n) \in C$ where $C = C', f, C''$ and $f$ turns into $f' = (\Delta_a, \Delta_{b_1}, p_2; \Delta_{b_2}; p; \Omega_1, ..., \Omega_k; \Xi_1, ..., \Xi_m; \Lambda_1, ..., \Lambda_m; \Upsilon_1, ..., \Upsilon_n)$ such that:
         \begin{enumerate}
            \item $\mo \Gamma; \Delta_c; \Xi_1, ..., \Xi_m, p_2, \Xi_c; \cdot; H; C''', f', C''; R \rightarrow \Xi'; \Delta'; \Gamma'$ (well-formed in relation to $T$) where $\Delta_c = (\Delta_1, \Delta_2, \Xi) - (\Xi_1, ..., \Xi_m, p_2, \Xi_c)$
         \end{enumerate}
         \item With backtracking to a persistent continuation frame: \\$\exists f = [\Gamma_1, p_2, \Gamma_2; \Delta_{c_1}, \Delta_{c_2}; \Xi_c; \bang p; \Omega_1, ..., \Omega_k; \Lambda_1, ..., \Lambda_m; \Upsilon_1, ..., \Upsilon_n] \in C$ where $C = C', f, C''$ and $f$ turns into $f' = [\Gamma_2; \Delta_{c_1}, \Delta_{c_2}; \Xi_1, ..., \Xi_m; \bang p; \Omega_1, ..., \Omega_k; \Lambda_1, ..., \Lambda_m; \Upsilon_1, ..., \Upsilon_n]$ such that:
         \begin{enumerate}
            \item $\mo \Gamma; \Delta_{c_1}; \Xi_1, ..., \Xi_m, \Delta_{c_2}; \cdot; H; C'', f', C''; R \rightarrow \Xi'; \Delta'; \Gamma'$ (well-formed in relation to $T$) where $\Delta_{c_1}, \Delta_{c_2} = (\Delta_1, \Delta_2, \Xi) - (\Xi_1, ..., \Xi_m)$
         \end{enumerate}
      \end{enumerate}
\end{enumerate}

If $\cont C; H; R; \Gamma \rightarrow \Xi'; \Delta'; \Gamma'$ and $C$ is well-formed in relation to $T$ then either:

\begin{enumerate}
   \item Match fails: \\ $\cont \cdot; H; R; \Gamma \rightarrow \Xi'; \Delta'; \Gamma'$
   \item Match succeeds: \\ $\mz \Delta_x \rightarrow A$ (where $\Delta_x \subseteq \Delta_N$) where one sub-case is true:
   \begin{enumerate}
      \item With backtracking to a linear continuation frame: \\ $\exists f = (\Delta_a; \Delta_{b_1}, p_2, \Delta_{b_2}; \Xi_1, ..., \Xi_m; p; \Omega_1, ..., \Omega_k; \Lambda_1, ..., \Lambda_m; \Upsilon_1, ..., \Upsilon_n) \in C$ where $C = C', f, C''$ and $f$ turns into $f' = (\Delta_a, \Delta_{b_1}, p_2; \Delta_{b_2}; p; \Omega_1, ..., \Omega_k; \Xi_1, ..., \Xi_m; \Lambda_1, ..., \Lambda_m; \Upsilon_1, ..., \Upsilon_n)$ such that:
      \begin{enumerate}
         \item $\mo \Gamma; \Delta_c; \Xi_1, ..., \Xi_m, p_2, \Xi_c; \cdot; H; C''', f', C''; R \rightarrow \Xi'; \Delta'; \Gamma'$ (well-formed in relation to $T$) where $\Delta_c = (\Delta_1, \Delta_2, \Xi) - (\Xi_1, ..., \Xi_m, p_2, \Xi_c)$ and $\Delta_x = \Xi_1, ..., \Xi_m, p_2, \Xi_c$
      \end{enumerate}
      \item With backtracking to a persistent continuation frame: \\$\exists f = [\Gamma_1, p_2, \Gamma_2; \Delta_{c_1}, \Delta_{c_2}; \Xi_c; \bang p; \Omega_1, ..., \Omega_k; \Lambda_1, ..., \Lambda_m; \Upsilon_1, ..., \Upsilon_n] \in C$ where $C = C', f, C''$ and $f$ turns into $f' = [\Gamma_2; \Delta_{c_1}, \Delta_{c_2}; \Xi_1, ..., \Xi_m; \bang p; \Omega_1, ..., \Omega_k; \Lambda_1, ..., \Lambda_m; \Upsilon_1, ..., \Upsilon_n]$ such that:
      \begin{enumerate}
         \item $\mo \Gamma; \Delta_{c_1}; \Xi_1, ..., \Xi_m, \Delta_{c_2}; \cdot; H; C'', f', C''; R \rightarrow \Xi'; \Delta'; \Gamma'$ (well-formed in relation to $T$) where $\Delta_{c_1}, \Delta_{c_2} = \Delta_1, \Delta_2, \Xi - (\Xi_1, ..., \Xi_m)$ and $\Delta_x = \Xi_1, ..., \Xi_m, \Delta_{c2}$
      \end{enumerate}
   \end{enumerate}
\end{enumerate}
\end{lemma}

\begin{proof}
   Proof by mutual induction. In $\mo$on the size of $\Omega$ and on $\contlld$, first on the size of the second argument of the frame ($\Delta''$ and $\Gamma''$) and then on the size of the stack $C$. When inverting the assumption, the well-formedness of the stack and match are proven straightforwardly using the well-formedness of the assumption and the match equivalence theorem.
\end{proof}

For the induction hypothesis to be applicable in in Lemma~\ref{thm:body_match} there must be
a relation between the judgments $\mo$and $\contlld$.
We can define a lexicographic ordering $A \prec B$, meaning that proposition $A$ has a smaller proof than proposition $B$ (potentially $A$ is sub-proof of $B$),
or alternatively, $A$ is "executed later" than $B$.
The specific ordering is as follows:

\begin{enumerate}
   \item $\cont C; H; R; \Gamma \rightarrow \Xi'; \Delta'; \Gamma' \prec \cont C', C; H; R; \Gamma \rightarrow \Xi'; \Delta'; \Gamma'$\\
   The continuation must use the top of the stack $C'$ before using $C$;
   \item $\cont C', (\Delta, \Delta_1; \Delta_2; \Xi; p; \Omega; \Lambda; \Upsilon), C; H; R; \Gamma \rightarrow \Xi'; \Delta'; \Gamma'$\\
   \hspace*{1cm}$\prec \cont C'', (\Delta; \Delta_1, \Delta_2; \Xi; p; \Omega; \Lambda; \Upsilon), C; H; R; \Gamma \rightarrow \Xi'; \Delta'; \Gamma'$\\
   A continuation frame with more candidates has more steps to do than a frame with less candidates;
   \item $\cont C', [\Gamma_1; \Delta; \Xi; \bang p; \Omega; \Lambda; \Upsilon], C; H; R; \Gamma \rightarrow \Xi'; \Delta'; \Gamma'$\\
   \hspace*{1cm} $\prec \cont C'', [\Gamma_2, \Gamma_3; \Delta; \Xi; \bang p; \Omega; \Lambda; \Upsilon], C; H; R; \Gamma \rightarrow \Xi'; \Delta'; \Gamma'$\\
      Same as the previous one;
   \item $\cont C; H; R; \Gamma \rightarrow \Xi'; \Delta'; \Gamma' \prec \mo \Gamma; \Delta; \Xi; \Omega; H; C', C; R \rightarrow \Xi'; \Delta'; \Gamma'$\\
   Same as (1);
   \item $\mo \Gamma; \Delta; \Xi; \Omega; H; C; R \rightarrow \Xi'; \Delta'; \Gamma' \prec \cont C', C; H; R; \Gamma \rightarrow \Xi'; \Delta'; \Gamma'$\\
   Same as the previous one;
   \item $\mo \Gamma; \Delta''; \Xi''; \Omega'; H; C', C; R \rightarrow \Xi'; \Delta'; \Gamma' \prec \mo \Gamma; \Delta; \Xi; \Omega; H; C; R \rightarrow \Xi'; \Delta'; \Gamma'$ as long as $\Omega' \prec \Omega$\\
   Adding continuation frames to the stack makes the proof smaller as long as $\Omega$ is also smaller; 
   \item $\mo \Gamma; \Delta; \Xi; \Omega; H; C', (\Delta, \Delta_1; \Delta_2; \Xi; p; \Omega; \Lambda; \Upsilon), C; R \rightarrow \Xi'; \Delta'$\\
   \hspace*{1cm} $\prec \mo \Gamma; \Delta''; \Xi''; \Omega'; C'', (\Delta; \Delta_1, \Delta_2; \Xi; p; \Omega; \Lambda; \Upsilon), C; R \rightarrow \Xi'; \Delta'; \Gamma'$\\
   Same as (2);
   \item $\mo \Gamma; \Delta; \Xi; \Omega; H; C', [\Gamma_1; \Delta; \Xi; \bang p; \Omega; \Lambda; \Upsilon], C; R \rightarrow \Xi'; \Delta'; \Gamma'$\\
   \hspace*{1cm} $\prec \mo \Gamma; \Delta''; \Xi''; \Omega'; C'', [\Gamma_2, \Gamma_3; \Delta; \Xi; \bang p; \Omega; \Lambda; \Upsilon], C; R \rightarrow \Xi'; \Delta'; \Gamma'$\\
   Same as (3).
\end{enumerate}


\subsection{Comprehension Lemma}
The soundness of rule derivation is done trivially except for comprehensions and aggregates. In this section, we present
the comprehension lemma that can be used trivially to prove the soundness of derivation.

The comprehension lemma is built from three results: (1) proving that matching the body of a comprehension is sound
in relation to HLD;
(2) proving that updating the continuation stacks makes them suitable for use in the next comprehension applications;
(3) proving that deriving the head of the comprehension is sound in relation to HLD;
(4) proving that we can apply as many comprehensions as the database allows.

\subsubsection{Comprehension Body Match Soundness}

Proving that matching the body of a comprehension is sound in relation to HLD follows the structure of the Lemma~\ref{thm:body_match}. The lemma uses mutual induction on the recursive judgments $\mc$and $\contc$and
considers the three possible results of matching: failure, success with no backtracking and success with backtracking.

\begin{lemma}[Comprehension body match]\label{thm:comprehension_body_match}
Given a comprehension body $A$, consider a triplet $T = A; \Gamma; \Delta_{N}$ and a context $\Delta_{N} = \Delta_1, \Delta_2, \Xi$.

If a match $\mc \Gamma; \Delta_1, \Delta_2; \Xi_N; \Gamma_{N1}; \Delta_{N1}; \Xi; \Omega; C; P; A \com B; \Omega_N; \Delta_N \rightarrow \Xi'; \Delta'; \Gamma'$ is well-formed in relation to $T$ then either:

   \begin{enumerate}
      \item Match fails: \\
      $\done \Gamma; \Delta_N; \Xi_N; \Gamma_{N1}; \Delta_{N1}; \Omega_N \rightarrow \Xi'; \Delta'; \Gamma'$
      
      \item Match succeeds: \\
      $\mz \Gamma; \Delta_x \rightarrow A$ (where $\Delta_x \subseteq \Delta_N$) and one of the three sub-cases is true:
      \begin{enumerate}
         \item No backtracking to frames of stack $C$ or $P$ ($C \neq \cdot$):\\
            $\mc \Gamma; \Delta_1; \Xi_N; \Gamma_{N1}; \Delta_{N1}; \Xi, \Delta_2; \cdot; C', C; P; A \com B; \Omega_N; \Delta_N \rightarrow \Xi'; \Delta'; \Gamma'$ (well-formed in relation to $T$) and $\Delta_x = \Delta_2$
         \item No backtracking to frames of stack $P$ ($C = \cdot$):\\
         $\mc \Gamma; \Delta_1; \Xi_N; \Gamma_{N1}; \Delta_{N1}; \Xi, \Delta_2; \cdot; C'; P', P; A \com B; \Omega_N; \Delta_N \rightarrow \Xi'; \Delta'; \Gamma'$ (well-formed in relation to $T$) and $\Delta_x = \Delta_2$
         
         \item With backtracking to a linear continuation frame in stack $C$ ($C \neq \cdot$):\\
         $\exists f = (\Delta_a; \Delta_{b_1}, p_2, \Delta_{b_2}; p; \Xi_1, ..., \Xi_m; \Omega_1, ..., \Omega_k; \Lambda_1, ..., \Lambda_m; \Upsilon_1, ..., \Upsilon_n) \in C$ where $C = C', f, C''$ that turns into $f' = (\Delta_a, \Delta_{b_1}, p_2; \Delta_{b_2}; p; \Xi_1, ..., \Xi_m; \Omega_1, ..., \Omega_k; \Lambda_1, ..., \Lambda_m; \Upsilon_1, ..., \Upsilon_n)$ such that:
         \begin{enumerate}
            \item $\mc \Gamma; \Delta_c; \Xi_N; \Gamma_{N1}; \Delta_{N1}; \Xi_1, ..., \Xi_m, p_2, \Xi_c; \cdot; C''', f', C''; P; A \com B; \Omega_N; \Delta_N \rightarrow \Xi'; \Delta'; \Gamma'$ (well-formed in relation to $T$) where $\Delta_c = (\Delta_1, \Delta_2, \Xi) - (\Xi_1, ..., \Xi_m, p_2, \Xi_c)$ and $\Delta_x = \Xi_1, ..., \Xi_m, p_2, \Xi_c$
         \end{enumerate}
            
         \item With backtracking to a persistent continuation frame in stack $C$ ($C \neq \cdot$):\\
         $\exists f = [\Gamma_1, p_2, \Gamma_2; \Delta_{c_1}, \Delta_{c_2}; \Xi_1, ..., \Xi_m; \bang p; \Omega_1, ..., \Omega_k; \Lambda_1, ..., \Lambda_m; \Upsilon_1, ..., \Upsilon_n] \in C$ where $C = C', f, C''$ that turns into $f' = [\Gamma_2; \Delta_{c_1}, \Delta_{c_2}; \Xi_1, ..., \Xi_m; \bang p; \Omega_1, ..., \Omega_k; \Lambda_1, ..., \Lambda_m; \Upsilon_1, ..., \Upsilon_n]$ such that:
         \begin{enumerate}
            \item $\mc \Gamma; \Delta_{c_1}; \Xi_N; \Gamma_{N1}; \Delta_{N1}; \Delta_{c_2}, \Xi_1, ..., \Xi_m; \cdot; C''', f', C''; P; A \com B; \Omega_N; \Delta_N \rightarrow \Xi'; \Delta'; \Gamma'$ (well-formed in relation to $T$)
            where $\Delta_{c_1}, \Delta_{c_2} = (\Delta_1, \Delta_2, \Xi) - (\Xi_1, ..., \Xi_m)$ and $\Delta_x = \Delta_{c_2}, \Xi_1, ..., \Xi_m$
         \end{enumerate}
            
         \item With backtracking to a persistent continuation frame in stack $P$ ($C = \cdot$):\\
         $\exists f = [\Gamma_1, p_2, \Gamma_2; \Delta_{c_1}, \Delta_{c_2}; \Xi_1, ..., \Xi_m; \bang p; \Omega_1, ..., \Omega_k; \Lambda_1, ..., \Lambda_m; \Upsilon_1, ..., \Upsilon_n] \in P$ where $P = P', f, P''$ that turns into $f' = [\Gamma_2; \Delta_{c_1}, \Delta_{c_2}; \Xi_1, ..., \Xi_m; \bang p; \Omega_1, ..., \Omega_k; \Lambda_1, ..., \Lambda_m; \Upsilon_1, ..., \Upsilon_n]$ such that:
            \begin{enumerate}
               \item $\mc \Gamma; \Delta_{c_1}; \Xi_N; \Gamma_{N1}; \Delta_{N1}; \Delta_{c_2}, \Xi_1, ..., \Xi_m; \cdot; C'; P''', f', P''; A \com B; \Omega_N; \Delta_N \rightarrow \Xi'; \Delta'; \Gamma'$ (well-formed in relation to $T$) where $\Delta_{c_1}, \Delta_{c_2} = (\Delta_1, \Delta_2, \Xi) - (\Xi_1, ..., \Xi_m)$ and $\Delta_x = \Delta_{c_2}, \Xi_1, ..., \Xi_m$
            \end{enumerate}
      \end{enumerate}
   \end{enumerate}
   
If $\contc \Gamma; \Delta_{N}; \Xi_{N}; \Gamma_{N1}; \Delta_{N1}; C; P; A \com B; \Omega_N \rightarrow \Xi'; \Delta'; \Gamma'$ and $C$ and $P$ are well-formed in relation to $T$ then either:

\begin{enumerate}
   \item Match fails: \\
   $\done \Gamma; \Delta_N; \Xi_N; \Gamma_{N1}; \Delta_{N1}; \Omega_N \rightarrow \Xi'; \Delta'; \Gamma'$
   
   \item Match succeeds: \\
   $\mz \Delta_x \rightarrow A$ (where $\Delta_x \subseteq \Delta_N$) where one sub-case is true:
   
   \begin{enumerate}
      \item With backtracking to a linear continuation frame in stack $C$ ($C \neq \cdot$):\\
      $\exists f = (\Delta_a; \Delta_{b_1}, p_2, \Delta_{b_2}; p; \Xi_1, ..., \Xi_m; \Omega_1, ..., \Omega_k; \Lambda_1, ..., \Lambda_m; \Upsilon_1, ..., \Upsilon_n) \in C$ where $C = C', f, C''$ that turns into $f' = (\Delta_a, \Delta_{b_1}, p_2; \Delta_{b_2}; p; \Xi_1, ..., \Xi_m; \Omega_1, ..., \Omega_k; \Lambda_1, ..., \Lambda_m; \Upsilon_1, ..., \Upsilon_n)$ such that:
         \begin{enumerate}
            \item $\mc \Gamma; \Delta_c; \Xi_N; \Gamma_{N1}; \Delta_{N1}; \Xi_1, ..., \Xi_m, p_2, \Xi_c; \cdot; C''', f', C''; P; A \com B; \Omega_N; \Delta_N \rightarrow \Xi'; \Delta'; \Gamma'$ (well-formed in relation to $T$) where $\Delta_c = (\Delta_1, \Delta_2, \Xi) - (\Xi_1, ..., \Xi_m, p_2, \Xi_c)$
         \end{enumerate}
      
      \item With backtracking to a persistent continuation frame in stack $C$ ($C \neq \cdot$):\\
      $\exists f = [\Gamma_1, p_2, \Gamma_2; \Delta_{c_1}, \Delta_{c_2}; \Xi_1, ..., \Xi_m; \bang p; \Omega_1, ..., \Omega_k; \Lambda_1, ..., \Lambda_m; \Upsilon_1, ..., \Upsilon_n] \in C$ where $C = C', f, C''$ that turns into $f' = [\Gamma_2; \Delta_{c_1}, \Delta_{c_2}; \Xi_1, ..., \Xi_m; \bang p; \Omega_1, ..., \Omega_k; \Lambda_1, ..., \Lambda_m; \Upsilon_1, ..., \Upsilon_n]$ such that:
         \begin{enumerate}
            \item $\mc \Gamma; \Delta_{c_1}; \Xi_N; \Gamma_{N1}; \Delta_{N1}; \Delta_{c_2}, \Xi_1, ..., \Xi_m; \cdot; C''', f', C''; P; A \com B; \Omega_N; \Delta_N \rightarrow \Xi'; \Delta'; \Gamma'$ (well-formed in relation to $T$)
            where $\Delta_{c_1}, \Delta_{c_2} = (\Delta_1, \Delta_2, \Xi) - (\Xi_1, ..., \Xi_m)$ and $\Delta_x = \Delta_{c_2}, \Xi_1, ..., \Xi_m$
         \end{enumerate}
         
      
      \item With backtracking to a persistent continuation frame in stack $P$ ($C = \cdot$):\\
      $\exists f = [\Gamma_1, p_2, \Gamma_2; \Delta_{c_1}, \Delta_{c_2}; \Xi_1, ..., \Xi_m; \bang p; \Omega_1, ..., \Omega_k; \Lambda_1, ..., \Lambda_m; \Upsilon_1, ..., \Upsilon_n] \in P$ where $P = P', f, P''$ that turns into $f' = [\Gamma_2; \Delta_{c_1}, \Delta_{c_2}; \Xi_1, ..., \Xi_m; \bang p; \Omega_1, ..., \Omega_k; \Lambda_1, ..., \Lambda_m; \Upsilon_1, ..., \Upsilon_n]$ such that:
         \begin{enumerate}
            \item $\mc \Gamma; \Delta_{c_1}; \Xi_N; \Gamma_{N1}; \Delta_{N1}; \Delta_{c_2}, \Xi_1, ..., \Xi_m; \cdot; C'; P''', f', P''; A \com B; \Omega_N; \Delta_N \rightarrow \Xi'; \Delta'; \Gamma'$ (well-formed in relation to $T$) where $\Delta_{c_1}, \Delta_{c_2} = (\Delta_1, \Delta_2, \Xi) - (\Xi_1, ..., \Xi_m)$ and $\Delta_x = \Delta_{c_2}, \Xi_1, ..., \Xi_m$
         \end{enumerate}
   \end{enumerate}
\end{enumerate}
\end{lemma}
\begin{proof}
Proof by mutual induction. See Lemma~\ref{thm:body_match} for details.
\end{proof}

\subsubsection{Stack Update}

In order to prove that both stacks $C$ and $P$ are usable for the next application of the iteration,
we need to prove that $C$ will have at most one updated linear continuation frame and $P$ will have
all its frames updated to account the consumption of the facts from the previous application of the
comprehension.

We now prove some auxiliary theorems.

\begin{theorem}[Full stack update]\label{thm:stack_update}
If $\strans \Xi; P; P'$ then $P'$ will be the transformation of stack $P$ where $\forall f = [\Gamma'; \Delta_N; \cdot; \bang p; \Omega; \cdot; \Upsilon] \in P$ will turn into $f' = [\Gamma'; \Delta_N - \Xi; \cdot; \bang p; \Omega; \cdot; \Upsilon]$.
\end{theorem}

\begin{proof}
Straightforward induction on the size of $P$.
\end{proof}

\begin{theorem}[From update to derivation]\label{thm:from_update_to_derivation}
   If $\dall \Gamma; \Xi_N; \Gamma_{N1}; \Delta_{N1}; \Xi; C; P; A \com B; \Omega_N; \Delta_N \rightarrow \Xi'; \Delta'; \Gamma'$ then
  $\dc \Gamma; \Xi_N, \Xi; \Gamma_{N1}; \Delta_{N1}; B; C' ; P'; A \com B; \Omega_N; (\Delta_N - \Xi) \rightarrow \Xi'; \Delta'; \Gamma'$, where:

\begin{enumerate}
   \item If $C = \cdot$ then $C' = \cdot$;
   \item If $C = C_1, (\Delta_a; \Delta_b; \cdot; p; \Omega; \cdot; \Upsilon)$ then $C' = (\Delta_a - \Xi; \Delta_b - \Xi; \cdot; p; \Omega; \cdot; \Upsilon)$;
   \item $P'$ is the transformation of stack $P$, where $\forall f = [\Gamma'; \Delta_N; \cdot; \bang p; \Omega; \cdot; \Upsilon] \in P$ will turn into \linebreak $f'=[\Gamma';\Delta_N-\Xi;\cdot;\bang p;\Omega;\cdot;\Upsilon]$.
\end{enumerate}
\end{theorem}

\begin{proof}
Use induction on the size of the stack $C$ to get the result of $C'$ then apply Theorem~\ref{thm:stack_update} to get $P'$.
\end{proof}

Now we prove that a match of a comprehension's body implies the start of a derivation of the comprehension's head with correct continuation stacks. Note that $\Omega = \cdot$ in $\matchlldc$, so there is nothing left to match.

\begin{lemma}[Match to derivation]\label{thm:match_to_derivation}
   If $\mc \Gamma; \Delta; \Xi_N; \Gamma_{N1}; \Delta_{N1}; \Xi; \cdot; B; C; P; A \com B;\Omega_N; \Delta_N \rightarrow \Xi'; \Delta'; \Gamma'$ then
      $\dc \Gamma; \Xi_N, \Xi; \Gamma_{N1}; \Delta_{N1}; B; C'; P'; A \com B; \Omega_N; (\Delta_N - \Xi) \rightarrow \Xi'; \Delta'; \Gamma'$ where:
   
   \begin{enumerate}
      \item If $C = \cdot$ then $C' = \cdot$;
      \item If $C = C_1, (\Delta_a; \Delta_b; \cdot; p; \Omega; \cdot; \Upsilon)$ then $C' = (\Delta_a - \Xi; \Delta_b - \Xi; \cdot; p; \Omega; \cdot; \Upsilon)$ then \linebreak $C' = (\Delta_a - \Xi; \Delta_b - \Xi; \cdot; p; \Omega; \cdot; \Upsilon)$;
      \item $P'$ is the transformation of stack $P$, where $\forall f = [\Gamma'; \Delta_N; \cdot; \bang p; \Omega; \cdot; \Upsilon] \in P$ will turn into \linebreak $f' = [\Gamma'; \Delta_N - \Xi; \cdot; \bang p; \Omega; \cdot; \Upsilon]$.
   \end{enumerate}
\end{lemma}

\begin{proof}
Invert the assumption and then apply Theorem~\ref{thm:from_update_to_derivation}.
\end{proof}

\subsubsection{Comprehension Derivation}

We also need to prove that deriving the head of a comprehension is sound in relation to HLD.
With the results of the next theorem we can reuse the continuation stacks to start the comprehension
process all over again, but now with a non-empty continuation stack.

\begin{theorem}[Comprehension derivation soundness]\label{thm:comprehension_derivation}
~\newline
If $\dc \Gamma; \Xi_N; \Gamma_{N1}; \Delta_{N1}; \Omega_1, ..., \Omega_n; C; P; A \com B; \Omega_N; \Delta_N \rightarrow \Xi'; \Delta'; \Gamma'$ then:

\begin{enumerate}
   \item $\dc \Gamma; \Xi_N; \Gamma_{N1}, \Gamma_1, ..., \Gamma_n; \Delta_{N1}, \Delta_1, ..., \Delta_n; \cdot; C; P; A \com B; \Omega_N; \Delta_N \rightarrow \Xi'; \Delta'; \Gamma'$;
   \item $\forall_{\Omega_x, \Delta_x}.$ If $\dz \Gamma; \Delta_x; \Xi_N; \Gamma_{N1}, \Gamma_1, ..., \Gamma_n; \Delta_{N1}, \Delta_1, ..., \Delta_n; \Omega_x \rightarrow \Xi'; \Delta'; \Gamma'$ then \linebreak $\dz \Gamma; \Delta_x; \Xi_N; \Gamma_{N1}; \Delta_{N1}; \Omega_1, ..., \Omega_n, \Omega_x \rightarrow \Xi'; \Delta'; \Gamma'$.
\end{enumerate}
\end{theorem}

\begin{proof}
Straightforward induction on $\Omega_1, ..., \Omega_n$.
\end{proof}

The second result of this theorem is the soundness result we need because it will allow us to reconstruct the derivation tree in HLD.

\subsubsection{Multiple Comprehension Derivation}

We are interested in proving that if we start with a given comprehension match $\matchlldc$ then we apply the comprehension several times.

\begin{theorem}[Multiple comprehension derivation]\label{thm:multiple_comprehension_derivation}
Assume that there exists $n \geq 0$ applications of a comprehension $A \com B$, where the $i_{th}$ application is related to the following contexts:
\begin{itemize}
   \item $\Delta_i$: context of derived linear facts;
   \item $\Gamma_i$: context of derived persistent facts;
   \item $\Xi_i$: context of consumed linear facts.
\end{itemize}
Consider a triplet $T = A; \Gamma; \Delta_{N}$, and $\Delta_N = \Delta, \Xi_1, ..., \Xi_n$.

Assume that $\Delta_N = \Delta_a, \Delta'_b, p_1$ and $\Delta_b = \Delta'_b, p_1$.
If $\mc \Gamma; \Delta_a, \Delta'_b; \Xi_N; \Gamma_{N1}; \Delta_{N1}; p_1; \Omega; (\Delta_a, p_1; \Delta'_b; \cdot; p; \Omega; \cdot; \Upsilon); P; A \com B; \Omega_N; \Delta, \Xi_1, ..., \Xi_n \rightarrow \Xi'; \Delta'; \Gamma'$ (well-formed in relation to $T$) and then:
   
   \begin{enumerate}
      \item $n$ comprehensions are derived:\\
      $\done \Gamma; \Delta_N; \Xi_N, \Xi_1, ..., \Xi_n; \Gamma_{N1}, \Gamma_1, ..., \Gamma_n; \Delta_{N1}, \Delta_1, ..., \Delta_n; \Omega_N \rightarrow \Xi'; \Delta'; \Gamma'$
      \item $n$ $\mz$propositions for the $n$ comprehension matches:\\
      $\mz \Gamma; \Xi_1 \rightarrow A$ ... $\mz \Gamma; \Xi_n \rightarrow A$
      \item $n$ implications from $1...i...n$ such that: \\
      $\forall_{\Omega_x, \Delta_x}.$ if $\done \Gamma; \Delta_x; \Xi_N, \Xi_1, ..., \Xi_i; \Gamma_{N1}, \Gamma_1, ..., \Gamma_i; \Delta_{N1}, \Delta_1, ..., \Delta_i; \Omega_x \rightarrow \Xi'; \Delta'; \Gamma'$ then \linebreak $\dz \Gamma; \Delta_x; \Xi_N, \Xi_1, ..., \Xi_i; \Gamma_{N1}, \Gamma_1, ..., \Gamma_{i-1}; \Delta_{N1}, \Delta_1, ..., \Delta_{i-1}; B, \Omega_x \rightarrow \Xi'; \Delta'; \Gamma'$
   \end{enumerate}
   
   If $\mc \Gamma; \Delta_N; \Xi_N; \Gamma_{N1}; \Delta_{N1}; \cdot; \Omega; \cdot; P; A \com B; \Omega_N; \Delta, \Xi_1, ..., \Xi_n \rightarrow \Xi'; \Delta'; \Gamma'$ (well-formed in relation to $T$) then:
   
   \begin{enumerate}
      \item $n$ comprehensions are derived:\\
      $\done \Gamma; \Delta_N; \Xi_N, \Xi_1, ..., \Xi_n; \Gamma_{N1}, \Gamma_1, ..., \Gamma_n; \Delta_{N1}, \Delta_1, ..., \Delta_n; \Omega_N \rightarrow \Xi'; \Delta'; \Gamma'$
      \item $n$ $\mz$propositions for the $n$ comprehension matches:\\
      $\mz \Gamma; \Xi_1 \rightarrow A$ ... $\mz \Gamma; \Xi_n \rightarrow A$
      \item $n$ implications from $1...i...n$ such that: \\
      $\forall_{\Omega_x, \Delta_x}.$ if $\dz \Gamma; \Delta_x; \Xi_N, \Xi_1, ..., \Xi_i; \Gamma_{N1}, \Gamma_1, ..., \Gamma_i; \Delta_{N1}, \Delta_1, ..., \Delta_i; \Omega_x \rightarrow \Xi'; \Delta'; \Gamma'$ then \\$\dz \Gamma; \Delta_x; \Xi_N, \Xi_1, ..., \Xi_i; ; \Gamma_{N1}, \Gamma_1, ..., \Gamma_{i-1}; \Delta_{N1}, \Delta_1, ..., \Delta_{i-1}; B, \Omega_x \rightarrow \Xi'; \Delta'; \Gamma'$
   \end{enumerate}
\end{theorem}

\begin{proof}
By mutual induction, first on either the size of $\Delta'_b$ (second argument of the linear continuation frame) or $\Gamma'$ (second argument of the persistent frame in $P$) and then on the size of $C, P$.

We only show how to prove the first implication since the second implication is proven in a similar way.

By applying the Comprehension body match lemma (Lemma~\ref{thm:comprehension_body_match}) to the assumption (1), we either get:

\begin{itemize}
   \item Failure:
   
   $\done \Gamma; \Delta_N; \Xi_N; \Gamma_{N1}; \Delta_{N1}; \Omega_N \rightarrow \Xi'; \Delta'; \Gamma'$ \hfill (2) from lemma (1) (so $n = 0$)\\
   
   \item Success:
   
   $\mz \Gamma; \Xi_1 \rightarrow A$ \hfill (2) from lemma (1) \\
   
   \begin{enumerate}
      \item No backtracking to frames of stack $C$ or $P$:\\
      {\footnotesize 
      $\Xi_1 = \Xi'_1, p_1$ \hfill (3) from the well-formedness of the assumption \\
      $\mc \Gamma; \Delta, \Xi_2, ..., \Xi_n; \Xi_N; \Gamma_{N1}; \Delta_{N1}; p_1, \Xi'_1; \cdot; C', (\Delta_a, p_1; \Delta'_b; \cdot; p; \Omega; \cdot; \Upsilon); P; A \com B; \Omega_N; \Delta_N \rightarrow \Xi'; \Delta'; \Gamma'$ \\ ... \hfill (4) from lemma (1) \\
      $\dc \Gamma; \Xi_N, \Xi_1; \Gamma_{N1}; \Delta_{N1}; B; (\Delta_a, p_1 - \Xi_1; \Delta_b - \Xi_1; \cdot; p; \Omega; \cdot; \Upsilon); P'; A \com B; \Omega_N; \Delta, \Xi_2, ..., \Xi_n \rightarrow \Xi'; \Delta'; \Gamma'$ \\ ... \hfill (5) using Match to derivation lemma (Lemma~\ref{thm:match_to_derivation}) to (4) \\
      $\dc \Gamma; \Xi_N, \Xi_1; \Gamma_{N1}, \Gamma_1; \Delta_{N1}, \Delta_1; \cdot; (\Delta_a - \Xi'_1; \Delta'_b - \Xi'_1; \cdot; p; \Omega; \cdot; \Upsilon); P; A \com B; \Omega_N; \Delta, \Xi_2, ..., \Xi_n \rightarrow \Xi'; \Delta'$ \\ ... \hfill (6) applying Comprehension derivation soundness theorem (Theorem~\ref{thm:comprehension_derivation}) on (5) \\
      if $\forall_{\Omega_x, \Delta_x}. \dz \Gamma; \Delta_x; \Xi_N, \Xi_1; \Gamma_{N1}, \Gamma_1; \Delta_{N1}, \Delta_1; \Omega_x \rightarrow \Xi'; \Delta'; \Gamma'$ then \\ \hspace*{0.5cm} $\dz \Gamma; \Delta_x; \Xi_N, \Xi_1; \Gamma_{N1}; \Delta_{N1}; B, \Omega_x \rightarrow \Xi'; \Delta'; \Gamma'$ \hfill (7) from theorem in (6) \\
      $\contc \Gamma; \Delta, \Xi_2, ..., \Xi_n; \Xi_N, \Xi_1; \Gamma_{N1}, \Gamma_1; \Delta_{N1}, \Delta_1; (\Delta_a - \Xi'_1; \Delta'_b - \Xi'_1; \cdot; p; \Omega; \cdot; \Upsilon); P'; A \com B; \Omega_N \rightarrow \Xi'; \Delta'; \Gamma'$ \\... \hfill (8) inversion of (6) \\
        
        By inverting (8) we either fail (thus $n = 1$) or we get a new match. For the latter case, we apply mutual induction to get the remaining $n - 1$ comprehensions.\\
      }
      
      \item With backtracking to the linear continuation frame of stack $C$:
      
      {\footnotesize
      
      $f = (\Delta_a, p_1; \Delta'_b; \cdot; p; \Omega; \cdot; \Upsilon)$ \hfill (3) frame to backtrack to \\
      turns into $f' = (\Delta_a, p_1, \Delta'''_b, p_2; \Delta''_b; \cdot; p; \Omega; \cdot; \Upsilon)$ \hfill (4) resulting frame \\
      $\mc \Gamma; \Delta, \Xi_2, ..., \Xi_n; \Xi_N; \Gamma_{N1}; \Delta_{N1}; p_2, \Xi'_1; \cdot; C', f'; P; A \com B; \Omega_N; \Delta_N \rightarrow \Xi'; \Delta'; \Gamma'$ \hfill (5) from lemma (1) \\
      
      Use the same approach as the case with no backtracking.\\
      
      }
      
      \item With backtracking to a persistent continuation frame of stack $P$:

      {\footnotesize 

      $f = [\Gamma''_1, p_2, \Gamma''_2; \Delta_N; \cdot; \bang p; \Omega; \cdot; \Upsilon]$ \hfill (4) from theorem \\
      turns into $f' = [\Gamma''_2; \Delta_N; \cdot; \bang p; \Omega; \cdot; \Upsilon]$ \hfill (5) from theorem \\
      $\mc \Gamma; \Delta, \Xi_2, ..., \Xi_n; \Xi_N; \Gamma_{N1}; \Delta_{N1}; \Xi_1; \cdot; C'; P', f', P''; A \com B; \Omega_N; \Delta_N \rightarrow \Xi'; \Delta'; \Gamma'$ \hfill (6) from theorem \\
         
      Use the same approach as the case with no backtracking.\\

      }
      
   \end{enumerate}
\end{itemize}
\end{proof}

With this theorem, we can finally prove the comprehension lemma, which gives us $n$ applications
of the comprehension starting from the beginning of the mechanism for deriving a comprehension.

\begin{lemma}[Comprehension]\label{thm:comprehension}
Assume that there exists $n \geq 0$ applications of a comprehension $A \com B$, where the $i_{th}$ application is related to the following contexts:
\begin{itemize}
   \item $\Delta_i$: context of derived linear facts;
   \item $\Gamma_i$: context of derived persistent facts;
   \item $\Xi_i$: context of consumed linear facts.
\end{itemize}
Consider a triplet $T = A; \Gamma; \Delta_{N}$, and $\Delta_N = \Delta, \Xi_1, ..., \Xi_n$.

If $\mc \Gamma; \Delta, \Xi_1, ..., \Xi_n; \Xi_N; \Gamma_{N1}; \Delta_{N1}; \cdot; A; \cdot; \cdot; A \com B; \Omega_N; \Delta, \Xi_1, ..., \Xi_n \rightarrow \Xi'; \Delta'; \Gamma'$ (well-formed in relation to $T$) then:

\begin{enumerate}
   \item $n$ comprehensions are derived:\\
   $\done \Gamma; \Delta_N; \Xi_N, \Xi_1, ..., \Xi_n; \Gamma_{N1}, \Gamma_1, ..., \Gamma_n; \Delta_{N1}, \Delta_1, ..., \Delta_n; \Omega_N \rightarrow \Xi'; \Delta'; \Gamma'$
   \item $n$ $\mz$propositions for the $n$ comprehension matches:\\
   $\mz \Gamma; \Xi_1 \rightarrow A$ ... $\mz \Gamma; \Xi_n \rightarrow A$
   \item $n$ implications from $1...i...n$ such that: \\
   $\forall_{\Omega_x, \Delta_x}.$ if $\dz \Gamma; \Delta_x; \Xi_N, \Xi_1, ..., \Xi_i; \Gamma_{N1}, \Gamma_1, ..., \Gamma_i; \Delta_{N1}, \Delta_1, ..., \Delta_i; \Omega_x \rightarrow \Xi'; \Delta'; \Gamma'$ then \linebreak $\dz \Gamma; \Delta_x; \Xi_N, \Xi_1, ..., \Xi_i; \Gamma_{N1}, \Gamma_1, ..., \Gamma_{i-1}; \Delta_{N1}, \Delta_1, ..., \Delta_{i-1}; B, \Omega_x \rightarrow \Xi'; \Delta'; \Gamma'$
\end{enumerate}
\end{lemma}

\begin{proof}
We apply the comprehension body match lemma (Lemma~\ref{thm:comprehension_body_match}) to get two sub-cases:
   
\begin{itemize}
   \item Match fails:
   
   {\small
   
   $\done \Gamma; \Delta_N; \Xi_N; \Gamma_{N1}; \Delta_{N1}; \Omega_N \rightarrow \Xi'; \Delta'; \Gamma'$ \hfill (1) no comprehension application was possible ($n = 0$)\\
   }
   
   \item Match succeeds:
   
   {\small
   $\mc \Gamma; \Xi_2, ..., \Xi_n; \Xi_N; \Gamma_{N1}; \Delta_{N1}; \Xi_1; \cdot; C; P; A \com B; \Omega_N; \Delta_N \rightarrow \Xi'; \Delta'; \Gamma'$
   
   ... \hfill (1) result from lemma~\ref{thm:comprehension_body_match}
   
   $\mz \Gamma; \Xi_1 \rightarrow A$
   
   ... \hfill (2) also from lemma~\ref{thm:comprehension_body_match}
   
   $\dc \Gamma; \Xi_N, \Xi_1; \Gamma_{N1}; \Delta_{N1}; B; C'; P'; A \com B; \Omega_N; \Delta, \Xi_2, ..., \Xi_n \rightarrow \Xi'; \Delta'; \Gamma'$
   
   ...\hfill (3) applying Lemma~\ref{thm:match_to_derivation}
   
   $\dc \Gamma; \Xi_N, \Xi_1; \Gamma_{N1}, \Gamma_1; \Delta_{N1}, \Delta_1; \cdot; C'; P'; A \com B; \Omega_N; \Delta, \Xi_2, ..., \Xi_n \rightarrow \Xi'; \Delta'; \Gamma'$
   
   ... \hfill (4) using Theorem~\ref{thm:comprehension_derivation} on (1)
   
   if $\forall_{\Omega_x, \Delta_x}. \dz \Gamma; \Delta_x; \Xi_N, \Xi_1; \Gamma_{N1}, \Gamma_1; \Delta_{N1}, \Delta_1; \Omega_x \rightarrow \Xi'; \Delta'; \Gamma'$ then
   
    \hspace*{0.5cm} $\dz \Gamma; \Delta_x; \Xi_N, \Xi_1; \Gamma_{N1}; \Delta_{N1}; B, \Omega_x \rightarrow \Xi'; \Delta'; \Gamma'$ \hfill (5) from the theorem applied in (4)
   
   $\contc \Gamma; \Delta, \Xi_2, ..., \Xi_n; \Xi_N, \Xi_1; \Gamma_{N1}, \Gamma_1; \Delta_{N1}, \Delta_1; C'; P'; A \com B; \Omega_N \rightarrow \Xi'; \Delta'; \Gamma'$
   
   ... \hfill (6) inversion of (5)
   
   Invert (6) to get either $n = 1$ application of the comprehension or apply Theorem~\ref{thm:multiple_comprehension_derivation} to the inversion to get the remaining $n-1$. 
   
   }
\end{itemize}
\end{proof}

\subsection{Head Derivation Soundness}
We are finally ready to prove that the derivation of terms of the head of a rule is sound in relation to HLD.

\begin{lemma}[Head derivation soundness]
\hspace{0.5cm}\\
If $\done \Gamma; \Delta_N; \Xi; \Gamma_1; \Delta_1; \Omega \rightarrow \Xi'; \Delta'; \Gamma'$ then $\dz \Gamma; \Delta_N; \Xi; \Gamma_1; \Delta_1; \Omega \rightarrow \Xi'; \Delta'; \Gamma'$.
\end{lemma}

\begin{proof}\label{sec:derivation_theorem}
Induction on $\Omega$. Most of the sub-cases can be proven using the induction hypothesis or by
straightforward rule inference. The sub-case for the comprehensions $A \com B$ is more
complicated and can be proved by using the comprehension lemma (Lemma~\ref{thm:comprehension}) on $\done \Gamma; \Delta; \Xi; \Gamma_1; \Delta_1; A \com B, \Omega \rightarrow \Xi'; \Delta'; \Gamma'$
 to get $n$ applications of the comprehension, where for the $i_{th}$ application we get the following:

\begin{itemize}
   \item $\Xi_{c_i}$: context of consumed linear facts ($\Delta_N = \Delta, \Xi_{c_1}, ..., \Xi_{c_n}$);
   \item $\Delta_{c_i}$: context of derived linear facts;
   \item $\Gamma_{c_i}$: context of derived persistent facts;
   \item if $\dz \Gamma; \Delta, \Xi_{c_n}, \Xi_{c_{i+1}}; \Xi, \Xi_{c_1}, ..., \Xi_{c_i}; \Gamma_{N1}, \Gamma_{c_1}, ..., \Gamma_{c_i}; \Delta_1, \Delta_{c_1}, ..., \Delta_{c_i}; A \com B, \Omega \rightarrow \Xi'; \Delta'; \Gamma'$ then \linebreak $\dz \Gamma; \Delta, \Xi_{c_n}, \Xi_{i_{n+1}}; \Xi, \Xi_{c_1}, ..., \Xi_{c_i}; \Gamma_1, \Gamma_{c_1}, ..., \Gamma_{c_{i-1}}; \Delta_1, \Delta_{c_1}, ..., \Delta_{c_{i-1}}; B, A \com B, \Omega \rightarrow \Xi'; \Delta'; \Gamma'$: the derivation implication.
\end{itemize}

$\done \Gamma; \Delta; \Xi, \Xi_{c_1}, ..., \Xi_{c_n}; \Gamma_1, \Gamma_{c_1}, ..., \Gamma_{c_n}; \Delta_1, \Delta_{c_1}, ..., \Delta_{c_n}; \Omega \rightarrow \Xi'; \Delta'; \Gamma'$ \hfill (1) from Lemma~\ref{thm:comprehension}

$\dz \Gamma; \Delta; \Xi, \Xi_{c_1}, ..., \Xi_{c_n}; \Gamma_1, \Gamma_{c_1}, ..., \Gamma_{c_n}; \Delta_1, \Delta_{c_1}, ..., \Delta_{c_n}; \Omega \rightarrow \Xi'; \Delta'; \Gamma'$ \hfill (2) i.h. on (1)

$\dz \Gamma; \Delta; \Xi, \Xi_{c_1}, ..., \Xi_{c_n}; \Gamma_1, \Gamma_{c_1}, ..., \Gamma_{c_n}; \Delta_1, \Delta_{c_1}, ..., \Delta_{c_n}; 1, \Omega \rightarrow \Xi'; \Delta'; \Gamma'$ \hfill (3) rule $\dz 1$ on (2)

$\dz \Gamma; \Delta; \Xi, \Xi_{c_1}, ..., \Xi_{c_n}; \Gamma_1, \Gamma_{c_1}, ..., \Gamma_{c_n}; \Delta_1, \Delta_{c_1}, ..., \Delta_{c_n}; 1 \with (A \lolli B \otimes A \com B), \Omega \rightarrow \Xi'; \Delta'; \Gamma'$

... \hfill (4) rule $\dz \with L$ on (3)


$\dz \Gamma; \Delta; \Xi, \Xi_{c_1}, ..., \Xi_{c_n}; \Gamma_1, \Gamma_{c_1}, ..., \Gamma_{c_n}; \Delta_1, \Delta_{c_1}, ..., \Delta_{c_n}; A \com B, \Omega \rightarrow \Xi'; \Delta'; \Gamma'$

... \hfill (5) rule $\dz\m{comp}$ on (4)

Using the $n^{th}$ implication of the comprehension lemma on (5):

$\dz \Gamma; \Delta; \Xi, \Xi_{c_1}, ..., \Xi_{c_n}; \Gamma_1, \Gamma_{c_1}, ..., \Gamma_{c_{n-1}}; \Delta_1, \Delta_{c_1}, ..., \Delta_{c_{n-1}}; B, A \com B, \Omega \rightarrow \Xi'; \Delta'; \Gamma'$

... \hfill (6) implication result

Using the $\mz \Xi_n \rightarrow A$ result on (6) with rule $\dz \lolli$:

$\dz \Gamma; \Delta, \Xi_{c_n}; \Xi, \Xi_{c_1}, ..., \Xi_{c_{n-1}}; \Gamma_1, \Gamma_{c_1}, ..., \Gamma_{c_{n-1}}; \Delta_1, \Delta_{c_1}, ..., \Delta_{c_{n-1}}; A \lolli B, A \com B, \Omega \rightarrow \Xi'; \Delta'; \Gamma'$ \hfill (7)

$\dz \Gamma; \Delta, \Xi_{c_n}; \Xi, \Xi_{c_1}, ..., \Xi_{c_{n-1}}; \Gamma_1, \Gamma_{c_1}, ..., \Gamma_{c_{n-1}}; \Delta_1, \Delta_{c_1}, ..., \Delta_{c_{n-1}}; A \lolli B \otimes A \com B, \Omega \rightarrow \Xi'; \Delta'; \Gamma'$

... \hfill (8) rule $\dz \otimes$ on (7)

$\dz \Gamma; \Delta, \Xi_{c_n}; \Xi, \Xi_{c_1}, ..., \Xi_{c_{n-1}}; \Gamma_1, \Gamma_{c_1}, ..., \Gamma_{c_{n-1}}; \Delta_1, \Delta_{c_1}, ..., \Delta_{c_{n-1}}; 1 \with (A \lolli B \otimes A \com B), \Omega \rightarrow \Xi'; \Delta'; \Gamma'$

... \hfill (9) rule $\dz \with R$ on (8)

$\dz \Gamma; \Delta, \Xi_{c_n}; \Xi, \Xi_{c_1}, ..., \Xi_{c_{n-1}}; \Gamma_1, \Gamma_{c_1}, ..., \Gamma_{c_{n-1}}; \Delta_1, \Delta_{c_1}, ..., \Delta_{c_{n-1}}; A \com B, \Omega \rightarrow \Xi'; \Delta'; \Gamma'$

... \hfill (10) rule $\dz \m{comp}$ on (9)

By recursively applying the other $n-1$ comprehensions on (10), we finally get:

$\dz \Gamma; \Delta, \Xi_{c_1}, ..., \Xi_{c_n}; \Delta_1; \Gamma_1; A \com B, \Omega \rightarrow \Xi'; \Delta'; \Gamma'$ \hfill (11)

\end{proof}
